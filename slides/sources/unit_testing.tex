\documentclass{article}


% XeLaTeX
\usepackage{xltxtra}
\usepackage{xunicode}
\usepackage{listings}
\usepackage[landscape]{geometry}

% Fonts
\setmainfont{Arial}
\newfontfamily\cyrillicfont{Arial}
\setmonofont{Courier New}
%\setmonofont{DejaVu Sans Mono}

% Lang
\usepackage{polyglossia}
\setmainlanguage{russian}
\setotherlanguage{english}
\usepackage[dvipsnames]{xcolor}
\include{pythonlisting}

\lstset{
	language=python,
	keywordstyle=\color{Emerald}, 
	commentstyle=\color{OliveGreen},
	stringstyle=\color{Bittersweet},
	tabsize=4,
	numbers=left,
	xleftmargin=10pt,
	morekeywords={with,as},
}

\begin{document}
\LARGE

%-------------------------------------------------------------------------------

\center{Юнит-тестирование. unittest}
\begin{itemize}
\item http://pycheesecake.org/wiki/PythonTestingToolsTaxonomy
\item \lstinline!assert condition!
\item unittest
\item Фильтрация тестов, автопоиск, etc
\item \lstinline!assertEqual!, \lstinline!assertIn!, \lstinline!assertRaises!, 
		\lstinline!assertAlmostEqual!, \lstinline!assertDictContainsSubset!
\item python -m unittest test\_module1 test\_module2
\item python -m unittest test\_module.TestClass
\item python -m unittest test\_module.TestClass.test\_method
\item python -m unittest discover project\_directory '*\_test.py'

\end{itemize}
\newpage
%-------------------------------------------------------------------------------

\center{Юнит-тестирование. unittest}
{
\LARGE \vspace{15pt}
\begin{lstlisting}
	import unittest

	class DefaultWidgetSizeTestCase(unittest.TestCase):
	    def setUp(self):
	        self.widget = Widget('The widget')

	    def runTest(self):
	    	assert self.widget.size() == (50, 50), "incorrect default size"
	        self.assertEqual(self.widget.size(), \
	        	(50, 50), "incorrect default size")

	    def tearDown(self):
	        self.widget.dispose()
	        self.widget = None
	
	if __name__ == '__main__':
    	unittest.main()
\end{lstlisting}
}
\newpage

%-------------------------------------------------------------------------------

\center{Юнит-тестирование. nose}
\begin{itemize}
\item Автопоиск и исполнение тестов, плагины 
		(coverage, исполнение тестов в отдельных процессах)
\item Юнит тесты в функциях
\end{itemize}

{
\LARGE \vspace{15pt}
\begin{lstlisting}
	from nose.tools import eq_

	# test_some.py
	def test1():
		"test, that 1 == 2"
		assert 1 == 2
		eq_(1, 2)

\end{lstlisting}
}
\newpage

%-------------------------------------------------------------------------------
\center{Юнит-тестирование. oktest}
\begin{itemize}

\item py.test (pytest) 
\item oktest 

\end{itemize}

{
\LARGE \vspace{15pt}
\begin{lstlisting}
	from oktest import ok

	def test_func():
		ok (s) == 'foo'
		ok (s) != 'foo'
		ok (n) > 0     
		ok (fn).raises(Error)
		ok ([]).is_a(list)
\end{lstlisting}
}

\newpage
%-------------------------------------------------------------------------------
\center{Юнит-тестирование. mock}

{
\LARGE \vspace{15pt}
\begin{lstlisting}
	import mock

	mock = mock.Mock()
	mock.method(1, 2, 3, test='wow')
	mock.called == True
	mock.method.assert_called_with(1, 2, 3, test='wow')
	
	attrs = {'method.return_value': 3, \
			 'other.side_effect': KeyError}
	mock.configure_mock(**attrs)
	mock.method() == 3
	mock.other() # KeyError raised
\end{lstlisting}
}

\newpage
%-------------------------------------------------------------------------------
\center{Юнит-тестирование. mock}
{
\LARGE \vspace{15pt}
\begin{lstlisting}
	from mock import patch

	class Class(object):
		def method(self):
			pass

	with patch('__main__.Class') as MockClass:
		instance = MockClass.return_value
		instance.method.return_value = 'foo'
		assert Class() is instance
		assert Class().method() == 'foo'

	@patch.object(SomeClass, 'class_method')
	def test(mock_method):
		SomeClass.class_method(3)
		mock_method.assert_called_with(3)
	
	test()
\end{lstlisting}
}

\newpage

%-------------------------------------------------------------------------------
\center{Юнит-тестирование. ludibrio}
{
\LARGE \vspace{15pt}
\begin{lstlisting}
	from ludibrio import Mock

	with Mock() as MySQLdb:
		con = MySQLdb.connect('server', 'user', 'XXXX')
		con.select_db('DB') >> None
		cursor = con.cursor()
		cursor.execute('select * from numbers') >> None
		cursor.fetchall() >> [1,2,3,4,5]

	con = MySQLdb.connect('server', 'user', 'XXXX')
	con.select_db('DB')
	cursor = con.cursor()
	cursor.execute('select * from numbers')
	cursor.fetchall() == [1, 2, 3, 4, 5]

	MySQLdb.validate()
\end{lstlisting}
}

\newpage

%-------------------------------------------------------------------------------
\end{document}








