\documentclass{article}

% XeLaTeX
\usepackage{xltxtra}
\usepackage{xunicode}
\usepackage{listings}
\usepackage[landscape]{geometry}

% Fonts
\setmainfont{DejaVu Sans} %{Arial}
\newfontfamily\cyrillicfont{Nimbus Roman No9 L} %{Arial}
\setmonofont{Courier New}
%\setmonofont{Ubuntu Mono}

%\setmonofont{DejaVu Sans Mono}

% Lang
\usepackage{polyglossia}
\setmainlanguage{russian}
\setotherlanguage{english}
\usepackage[dvipsnames,table]{xcolor}


\ifx\pdfoutput\undefined
\usepackage{graphicx}
\else
\usepackage[pdftex]{graphicx}
\fi

\lstset{
	language=python,
	keywordstyle=\color{Emerald},%\texttt, 
	commentstyle=\color{OliveGreen},%\texttt,
	stringstyle=\color{Bittersweet},%\texttt,
	tabsize=4,
	numbers=left,
	xleftmargin=10pt,
	morekeywords={with,as},	
	numberstyle=\large,
	%identifierstyle=\texttt,
	%basicstyle=\texttt,
}

\usepackage{hyperref}

\hypersetup{
	colorlinks=true,
	urlcolor=blue
}

\usepackage{float}
%\floatstyle{boxed} 
%\restylefloat{figure}
\usepackage[normalem]{ulem}


\begin{document}
\LARGE

%--------------------------------------------------------------------
\center{Стандартная библиотека: сокеты}
\begin{itemize}
\item \lstinline!socket! - низкоуровневая работа с сетью
\end{itemize}
{
\Large \vspace{15pt}
\begin{lstlisting}
	def client(host, port=12000):
		# tcp сокет
		sock = socket.socket(socket.AF_INET, socket.SOCK_STREAM)
		# socket.SOCK_DGRAM - udp
		sock.connect((host, port))
		sock.send("HELLO")
		print sock.recv()

	def server(host='0.0.0.0', port=12000):
		sock = socket.socket(socket.AF_INET, socket.SOCK_STREAM)
		sock.bind((host, port))
		sock.listen(5)

		while True:
			cs, client_addr_info = sock.accept()
			print "connection from ", client_addr_info
			cs.recv(5)
			cs.send("BYE!")
			cs.close()
\end{lstlisting}
}
\newpage

%--------------------------------------------------------------------
\center{Стандартная библиотека: сокеты}
\begin{itemize}
\item \lstinline!gethostbyname!, \lstinline!socket.gethostbyname_ex! - DNS запрос
\item \lstinline!gethostbyaddr! - reverse DNS
\item \lstinline!setdefaulttimeout(timeout)! - установка таймаута для операциях на сокетах
\item \lstinline!htonl! - преобразование порядка байтов,.....
\item \lstinline!ssl! - ssl сокеты
\end{itemize}

{
\vspace{15pt}
\begin{lstlisting}
	import socket
	
	print socket.gethostbyaddr('8.8.8.8')
	# ('google-public-dns-a.google.com', [], ['8.8.8.8'])

	print socket.gethostbyname_ex("www.google.com")
	# ('www.l.google.com', ['www.google.com'], 
	#	['173.194.35.145', '173.194.35.146', '173.194.35.147', 
	#            '173.194.35.148', '173.194.35.144'])

\end{lstlisting}
}

\newpage

%--------------------------------------------------------------------
\center{Стандартная библиотека: сокеты}
\begin{itemize}
\item \lstinline!select! - асинхронная работа с сокетами
\item \lstinline!select.select(read_list, write_list, error_list, timeout=None)!
\item \lstinline!select.pool!, \lstinline!select.epool! - быстрее на больших списках дескрипторов
\end{itemize}

{
\vspace{15pt}
\begin{lstlisting}
	import select
	
	sc1 = socket.socket(socket.AF_INET, socket.SOCK_STREAM)
	sc2 = socket.socket(socket.AF_INET, socket.SOCK_STREAM)	
	sc1.connect(("www.google.com", 80))
	sc2.connect(("www.google.com", 80))

	r,w,e = select.select([sc1, sc2], [], [], 0.1)

	r == []
	w == []
	e == []
\end{lstlisting}
}

\newpage

%--------------------------------------------------------------------
\center{Другие сетевые библиотеки}
\begin{itemize}

\item scapy - ICMP, ARP, манипуляция сетевыми пакетами, etc
\item twisted - асинхроннай фреймворк, поддерживает большую 
		часть используемых сетевых протоколов
\item gevent - асинхронный фреймворк, поддерживает программирование без callback,
				эмулирует потоки


\end{itemize}
\newpage

%--------------------------------------------------------------------
\center{twisted}
{
\vspace{15pt}
\begin{lstlisting}
	from twisted.web import server, resource
	from twisted.internet import reactor

	class HelloResource(resource.Resource):
	    isLeaf = True
	    numberRequests = 0
	    
	    def render_GET(self, request):
	        self.numberRequests += 1
	        request.setHeader("content-type", "text/plain")
	        return "I am request #" + str(self.numberRequests) + "\n"

	reactor.listenTCP(8080, server.Site(HelloResource()))
	reactor.run()
\end{lstlisting}
}

\newpage

%--------------------------------------------------------------------
\center{HTTP}
\begin{itemize}
\item Стандартная библиотека: httplib, urllib, urllib2
{
\vspace{15pt}
\begin{lstlisting}
	import urllib2
	url = "http://search.yahoo.com/search?p=test"
	search_res = urllib2.urlopen(url).read()
\end{lstlisting}
}
\item urllib3
\item requests
\end{itemize}

\newpage

%--------------------------------------------------------------------
\center{Стандартная библиотека: Другие протоколы}
\begin{itemize}
\item poplib
\item imaplib
\item ftplib
\item telnetlib
\item BaseHttpServer
\item ...
\end{itemize}

\newpage

%--------------------------------------------------------------------
\end{document}