% python classes slides - functions
% (c) 2012 Kostiantyn Danylov aka koder 
% koder.mail@gmail.com
% distributed under CC-BY licence
% http://creativecommons.org/licenses/by/3.0/deed.en

\documentclass{article}
% XeLaTeX
\usepackage{xltxtra}
\usepackage{xunicode}
\usepackage{listings}
\usepackage[landscape]{geometry}

% Fonts
\setmainfont{DejaVu Sans} %{Arial}
\newfontfamily\cyrillicfont{Nimbus Roman No9 L} %{Arial}
\setmonofont{Courier New}
%\setmonofont{Ubuntu Mono}

%\setmonofont{DejaVu Sans Mono}

% Lang
\usepackage{polyglossia}
\setmainlanguage{russian}
\setotherlanguage{english}
\usepackage[dvipsnames,table]{xcolor}


\ifx\pdfoutput\undefined
\usepackage{graphicx}
\else
\usepackage[pdftex]{graphicx}
\fi

\lstset{
	language=python,
	keywordstyle=\color{Emerald},%\texttt, 
	commentstyle=\color{OliveGreen},%\texttt,
	stringstyle=\color{Bittersweet},%\texttt,
	tabsize=4,
	numbers=left,
	xleftmargin=10pt,
	morekeywords={with,as},	
	numberstyle=\large,
	%identifierstyle=\texttt,
	%basicstyle=\texttt,
}

\usepackage{hyperref}

\hypersetup{
	colorlinks=true,
	urlcolor=blue
}

\usepackage{float}
%\floatstyle{boxed} 
%\restylefloat{figure}
\usepackage[normalem]{ulem}


\makeatletter
\def\PY@reset{\let\PY@it=\relax \let\PY@bf=\relax%
    \let\PY@ul=\relax \let\PY@tc=\relax%
    \let\PY@bc=\relax \let\PY@ff=\relax}
\def\PY@tok#1{\csname PY@tok@#1\endcsname}
\def\PY@toks#1+{\ifx\relax#1\empty\else%
    \PY@tok{#1}\expandafter\PY@toks\fi}
\def\PY@do#1{\PY@bc{\PY@tc{\PY@ul{%
    \PY@it{\PY@bf{\PY@ff{#1}}}}}}}
\def\PY#1#2{\PY@reset\PY@toks#1+\relax+\PY@do{#2}}

\expandafter\def\csname PY@tok@gd\endcsname{\def\PY@tc##1{\textcolor[rgb]{0.63,0.00,0.00}{##1}}}
\expandafter\def\csname PY@tok@gu\endcsname{\let\PY@bf=\textbf\def\PY@tc##1{\textcolor[rgb]{0.50,0.00,0.50}{##1}}}
\expandafter\def\csname PY@tok@gt\endcsname{\def\PY@tc##1{\textcolor[rgb]{0.00,0.25,0.82}{##1}}}
\expandafter\def\csname PY@tok@gs\endcsname{\let\PY@bf=\textbf}
\expandafter\def\csname PY@tok@gr\endcsname{\def\PY@tc##1{\textcolor[rgb]{1.00,0.00,0.00}{##1}}}
\expandafter\def\csname PY@tok@cm\endcsname{\let\PY@it=\textit\def\PY@tc##1{\textcolor[rgb]{0.25,0.50,0.50}{##1}}}
\expandafter\def\csname PY@tok@vg\endcsname{\def\PY@tc##1{\textcolor[rgb]{0.10,0.09,0.49}{##1}}}
\expandafter\def\csname PY@tok@m\endcsname{\def\PY@tc##1{\textcolor[rgb]{0.40,0.40,0.40}{##1}}}
\expandafter\def\csname PY@tok@mh\endcsname{\def\PY@tc##1{\textcolor[rgb]{0.40,0.40,0.40}{##1}}}
\expandafter\def\csname PY@tok@go\endcsname{\def\PY@tc##1{\textcolor[rgb]{0.50,0.50,0.50}{##1}}}
\expandafter\def\csname PY@tok@ge\endcsname{\let\PY@it=\textit}
\expandafter\def\csname PY@tok@vc\endcsname{\def\PY@tc##1{\textcolor[rgb]{0.10,0.09,0.49}{##1}}}
\expandafter\def\csname PY@tok@il\endcsname{\def\PY@tc##1{\textcolor[rgb]{0.40,0.40,0.40}{##1}}}
\expandafter\def\csname PY@tok@cs\endcsname{\let\PY@it=\textit\def\PY@tc##1{\textcolor[rgb]{0.25,0.50,0.50}{##1}}}
\expandafter\def\csname PY@tok@cp\endcsname{\def\PY@tc##1{\textcolor[rgb]{0.74,0.48,0.00}{##1}}}
\expandafter\def\csname PY@tok@gi\endcsname{\def\PY@tc##1{\textcolor[rgb]{0.00,0.63,0.00}{##1}}}
\expandafter\def\csname PY@tok@gh\endcsname{\let\PY@bf=\textbf\def\PY@tc##1{\textcolor[rgb]{0.00,0.00,0.50}{##1}}}
\expandafter\def\csname PY@tok@ni\endcsname{\let\PY@bf=\textbf\def\PY@tc##1{\textcolor[rgb]{0.60,0.60,0.60}{##1}}}
\expandafter\def\csname PY@tok@nl\endcsname{\def\PY@tc##1{\textcolor[rgb]{0.63,0.63,0.00}{##1}}}
\expandafter\def\csname PY@tok@nn\endcsname{\let\PY@bf=\textbf\def\PY@tc##1{\textcolor[rgb]{0.00,0.00,1.00}{##1}}}
\expandafter\def\csname PY@tok@no\endcsname{\def\PY@tc##1{\textcolor[rgb]{0.53,0.00,0.00}{##1}}}
\expandafter\def\csname PY@tok@na\endcsname{\def\PY@tc##1{\textcolor[rgb]{0.49,0.56,0.16}{##1}}}
\expandafter\def\csname PY@tok@nb\endcsname{\def\PY@tc##1{\textcolor[rgb]{0.00,0.50,0.00}{##1}}}
\expandafter\def\csname PY@tok@nc\endcsname{\let\PY@bf=\textbf\def\PY@tc##1{\textcolor[rgb]{0.00,0.00,1.00}{##1}}}
\expandafter\def\csname PY@tok@nd\endcsname{\def\PY@tc##1{\textcolor[rgb]{0.67,0.13,1.00}{##1}}}
\expandafter\def\csname PY@tok@ne\endcsname{\let\PY@bf=\textbf\def\PY@tc##1{\textcolor[rgb]{0.82,0.25,0.23}{##1}}}
\expandafter\def\csname PY@tok@nf\endcsname{\def\PY@tc##1{\textcolor[rgb]{0.00,0.00,1.00}{##1}}}
\expandafter\def\csname PY@tok@si\endcsname{\let\PY@bf=\textbf\def\PY@tc##1{\textcolor[rgb]{0.73,0.40,0.53}{##1}}}
\expandafter\def\csname PY@tok@s2\endcsname{\def\PY@tc##1{\textcolor[rgb]{0.73,0.13,0.13}{##1}}}
\expandafter\def\csname PY@tok@vi\endcsname{\def\PY@tc##1{\textcolor[rgb]{0.10,0.09,0.49}{##1}}}
\expandafter\def\csname PY@tok@nt\endcsname{\let\PY@bf=\textbf\def\PY@tc##1{\textcolor[rgb]{0.00,0.50,0.00}{##1}}}
\expandafter\def\csname PY@tok@nv\endcsname{\def\PY@tc##1{\textcolor[rgb]{0.10,0.09,0.49}{##1}}}
\expandafter\def\csname PY@tok@s1\endcsname{\def\PY@tc##1{\textcolor[rgb]{0.73,0.13,0.13}{##1}}}
\expandafter\def\csname PY@tok@sh\endcsname{\def\PY@tc##1{\textcolor[rgb]{0.73,0.13,0.13}{##1}}}
\expandafter\def\csname PY@tok@sc\endcsname{\def\PY@tc##1{\textcolor[rgb]{0.73,0.13,0.13}{##1}}}
\expandafter\def\csname PY@tok@sx\endcsname{\def\PY@tc##1{\textcolor[rgb]{0.00,0.50,0.00}{##1}}}
\expandafter\def\csname PY@tok@bp\endcsname{\def\PY@tc##1{\textcolor[rgb]{0.00,0.50,0.00}{##1}}}
\expandafter\def\csname PY@tok@c1\endcsname{\let\PY@it=\textit\def\PY@tc##1{\textcolor[rgb]{0.25,0.50,0.50}{##1}}}
\expandafter\def\csname PY@tok@kc\endcsname{\let\PY@bf=\textbf\def\PY@tc##1{\textcolor[rgb]{0.00,0.50,0.00}{##1}}}
\expandafter\def\csname PY@tok@c\endcsname{\let\PY@it=\textit\def\PY@tc##1{\textcolor[rgb]{0.25,0.50,0.50}{##1}}}
\expandafter\def\csname PY@tok@mf\endcsname{\def\PY@tc##1{\textcolor[rgb]{0.40,0.40,0.40}{##1}}}
\expandafter\def\csname PY@tok@err\endcsname{\def\PY@bc##1{\setlength{\fboxsep}{0pt}\fcolorbox[rgb]{1.00,0.00,0.00}{1,1,1}{\strut ##1}}}
\expandafter\def\csname PY@tok@kd\endcsname{\let\PY@bf=\textbf\def\PY@tc##1{\textcolor[rgb]{0.00,0.50,0.00}{##1}}}
\expandafter\def\csname PY@tok@ss\endcsname{\def\PY@tc##1{\textcolor[rgb]{0.10,0.09,0.49}{##1}}}
\expandafter\def\csname PY@tok@sr\endcsname{\def\PY@tc##1{\textcolor[rgb]{0.73,0.40,0.53}{##1}}}
\expandafter\def\csname PY@tok@mo\endcsname{\def\PY@tc##1{\textcolor[rgb]{0.40,0.40,0.40}{##1}}}
\expandafter\def\csname PY@tok@kn\endcsname{\let\PY@bf=\textbf\def\PY@tc##1{\textcolor[rgb]{0.00,0.50,0.00}{##1}}}
\expandafter\def\csname PY@tok@mi\endcsname{\def\PY@tc##1{\textcolor[rgb]{0.40,0.40,0.40}{##1}}}
\expandafter\def\csname PY@tok@gp\endcsname{\let\PY@bf=\textbf\def\PY@tc##1{\textcolor[rgb]{0.00,0.00,0.50}{##1}}}
\expandafter\def\csname PY@tok@o\endcsname{\def\PY@tc##1{\textcolor[rgb]{0.40,0.40,0.40}{##1}}}
\expandafter\def\csname PY@tok@kr\endcsname{\let\PY@bf=\textbf\def\PY@tc##1{\textcolor[rgb]{0.00,0.50,0.00}{##1}}}
\expandafter\def\csname PY@tok@s\endcsname{\def\PY@tc##1{\textcolor[rgb]{0.73,0.13,0.13}{##1}}}
\expandafter\def\csname PY@tok@kp\endcsname{\def\PY@tc##1{\textcolor[rgb]{0.00,0.50,0.00}{##1}}}
\expandafter\def\csname PY@tok@w\endcsname{\def\PY@tc##1{\textcolor[rgb]{0.73,0.73,0.73}{##1}}}
\expandafter\def\csname PY@tok@kt\endcsname{\def\PY@tc##1{\textcolor[rgb]{0.69,0.00,0.25}{##1}}}
\expandafter\def\csname PY@tok@ow\endcsname{\let\PY@bf=\textbf\def\PY@tc##1{\textcolor[rgb]{0.67,0.13,1.00}{##1}}}
\expandafter\def\csname PY@tok@sb\endcsname{\def\PY@tc##1{\textcolor[rgb]{0.73,0.13,0.13}{##1}}}
\expandafter\def\csname PY@tok@k\endcsname{\let\PY@bf=\textbf\def\PY@tc##1{\textcolor[rgb]{0.00,0.50,0.00}{##1}}}
\expandafter\def\csname PY@tok@se\endcsname{\let\PY@bf=\textbf\def\PY@tc##1{\textcolor[rgb]{0.73,0.40,0.13}{##1}}}
\expandafter\def\csname PY@tok@sd\endcsname{\let\PY@it=\textit\def\PY@tc##1{\textcolor[rgb]{0.73,0.13,0.13}{##1}}}

\def\PYZbs{\char`\\}
\def\PYZus{\char`\_}
\def\PYZob{\char`\{}
\def\PYZcb{\char`\}}
\def\PYZca{\char`\^}
\def\PYZam{\char`\&}
\def\PYZlt{\char`\<}
\def\PYZgt{\char`\>}
\def\PYZsh{\char`\#}
\def\PYZpc{\char`\%}
\def\PYZdl{\char`\$}
\def\PYZti{\char`\~}
% for compatibility with earlier versions
\def\PYZat{@}
\def\PYZlb{[}
\def\PYZrb{]}
\makeatother

\begin{document}
\LARGE

%-------------------------------------------------------------------------------
\begin{center}Функции\end{center}
\begin{itemize}
    \item Именованный изолированный участок кода,
        принимающий параметры и возвращающий результат
    \item Все имена уничтожаются после выхода из функции
    \item Функция "не видит" имена, определенные в других 
        функциях (кроме случая вложенных функций)
    \item \lstinline!def name(param1, param2, ...)!
    \item \lstinline!return something! – возвращает результат
    \item \lstinline!return! - возвращает None
\end{itemize}
\newpage

%-------------------------------------------------------------------------------
\begin{center} Функции: области видимости \end{center}
\begin{lstlisting}
    def add(param1, param2):
        "returns sum of two objects"
        result = param1 + param2
        return result

    print add(2, 3) # 5
    print result # error
    print add(2, "3") # error
    print add("2", "3") #"23"

    help(add)
    
    # Help on function add in module __main__:
    #
    #add(param1, param2)
    #    returns sum of two objects

    print add.__doc__ # returns sum of two objects
\end{lstlisting}
\newpage

%-------------------------------------------------------------------------------
\begin{center} Функции: области видимости \end{center}
\begin{itemize}
    \item Все имена в функции при компиляции делатся на глобальные и локальные.
    \item Параметры и имена, которым где-либо производится присваивание, - локальные.
          Остальные - глобальные. На основании этого производится поиск во время исполнения.
\end{itemize}
\begin{lstlisting}
    def add_gv(val):
        return val + GLOBAL_VAR

    add_gv(1) # error
    GLOBAL_VAR = 12
    add_gv(10) == 22

    def add_gv_second(val):
        res = val + GLOBAL_VAR
        GLOBAL_VAR = val
        return res
    add_gv_second(10) # error
\end{lstlisting}
\newpage

%-------------------------------------------------------------------------------
\begin{center} Функции: глобальные переменные \end{center}
\vspace{15pt}
\begin{lstlisting}
    def never_do_this(value):
        global SOME_GLOBAL_VAR
        SOME_GLOBAL_VAR = value

    print SOME_GLOBAL_VAR # error
    never_do_this(1)
    print SOME_GLOBAL_VAR # 1
\end{lstlisting}
\newpage

%-------------------------------------------------------------------------------
\begin{center} Функции: значения по умолчанию \end{center}
\vspace{15pt}
\begin{lstlisting}
    def f1(v1, v2=0):
        print v1, v2

    f1(1) # 1 0
    f1(2, 3) # 2 3

    def f2(a, b=12, c):
        pass   #error

    def f3(v1, v2=[]):
        v2.append(v1)
        return v2

    f3(0) == [0]
    f3(1) == [0, 1] # O_o
\end{lstlisting}
\newpage

%-------------------------------------------------------------------------------
\begin{center} Функции: передача параметров \end{center}
\vspace{15pt}
\begin{lstlisting}
    def f3(a, *args):
        print a, args

    print f3(1) # 1 ()
    print f3(1, 2) # 1 (2,)
    print f3(1, 2, "abc") # 1 (2, "abc")

    def x(a, b):
        return a – b

    x(2, 1) == 1
    x(b=2, a=1) == -1
    x(1, a=1) # error
    x(a=1, 2) # error
\end{lstlisting}
\newpage

%-------------------------------------------------------------------------------
\begin{center} Функции: передача параметров \end{center}
\vspace{15pt}
\begin{lstlisting}
    def x(a, **kwargs):
        print a, kwargs

    x(2) => 2, {}
    x(2, 1) => error
    x(a=1, b=2, c=3) => 1, {"c":3, "b":2}
    x(1, fff=True) => 1, {"fff":True}

    def x(a, b, *args, **kwargs):
        pass
\end{lstlisting}
\newpage

%-------------------------------------------------------------------------------
\begin{center} Функции: передача параметров \end{center}
\vspace{15pt}
\begin{lstlisting}
    def x(a, b):
        print a, b

    params_tuple = (1, 2)
    x(*params_tuple) # 1 2

    params_dict = {"a":44, "b":33}
    x(**params_dict) # 44 33

    x(*[1], **{"b":2}) # 1 2
\end{lstlisting}
\newpage

%-------------------------------------------------------------------------------
\begin{center} Функции: python3.X \end{center}
\vspace{15pt}
Аннотации - связывание данных с параметрами, никак не используется языком
\begin{lstlisting}
    def x(a:int, b:12):
        print a, b
    print x.__annotation__ # {'a':int, 'b':12}
\end{lstlisting}
Только именованные аргументы
\begin{lstlisting}
    def x(a, b, *, m=None):
        pass
    x(1,2,3) # error
\end{lstlisting}
\newpage

%-------------------------------------------------------------------------------
\begin{center} lambda \end{center}
\begin{itemize}
    \item Способ сделать функцию из выражения
    \item \lstinline!func = lambda x : x + 2!
    \item lambda params : expression 
\end{itemize}
\begin{lstlisting}
def func(params):
    return expression
\end{lstlisting}
\begin{itemize}
    \item в теле lambda нельзя использовать конструкции
    \item но можно inline if, так как это выражение
\end{itemize}
\newpage 

%-------------------------------------------------------------------------------
\begin{center} ФП - Введение \end{center}
\begin{itemize}
    \item Имя функции – переменная, указывающая на объект-функцию в памяти
\end{itemize}
\vspace{15pt}
\begin{lstlisting}
    def sum(x, y):
        return x + y

    print sum # <function sum at 0x..>
    sum2 = sum
    sum2(1, 2) == 3
    sum2.func_name == 'sum'

    sum = FunctionType(CodeType(...), globals(), 'sum', ...)
\end{lstlisting}
\newpage

%-------------------------------------------------------------------------------
\begin{center} ФП - Введение \end{center}
\vspace{15pt}
\begin{lstlisting}
    def map(func, iter):
        res = []
        for i in iter:
            res.append(func(i))
        return res

    def mul2(x):
        return x * 2

    map(mul2, (1, 2, 3)) == [2, 4, 6]
    map(str, (1, 2, 3)) == ["1", "2", "3"]
\end{lstlisting}
\newpage

%-------------------------------------------------------------------------------
%level=3
\begin{center} ФП:Вложенные функции \end{center}
\begin{itemize}
    \item Новая функция создается каждый раз, когда python исполняет конструкцию def
    \item Функции могут бы вложенными, можно вернуть вложенную функцию
    \item Вложенная функция имеет доступ к аргументам функции, в которую она вложенна
\end{itemize}
\vspace{15pt}
\begin{lstlisting}
    def top_func(x):
        def embedded_func(val):
            return val * 2
        return embedded_func
\end{lstlisting}
\newpage

%-------------------------------------------------------------------------------
%level=3
\begin{center} ФП:Замыкания \end{center}
\begin{itemize}
    \item При возврате вложенной функции, использующей переменные родительской,
        она сохраняет ссылки на используемын переменные и не дает им уничтожиться
\end{itemize}
\vspace{15pt}
\begin{lstlisting}
    def add_some(x):
        def add_closure(val):
            return val + x
        return add_closure

    add1 = add_some(1)
    add5 = add_some(5)

    add1 is add5 == False

    add1(10) == 11
    add5(10) == 15

    map(add_some(10), (1, 2, 3)) == [11, 12, 13]
\end{lstlisting}
\newpage

%-------------------------------------------------------------------------------
%level=3
\begin{center} ФП:Каррирование \end{center}
\begin{itemize}
    \item Каррирование - связывание функции с частью параметров, с созданием новой функции
\end{itemize}
\vspace{15pt}
\begin{lstlisting}
    def bind_1st(func, val):
        def closure(*vals, **params):
            return func(val, *vals, **params)
        return closure

    def max(x, y):
         if y > x:
            return y
        return x

    not_less_then_10 = bind_1st(max, 10)
    map(not_less_then_10, (0, 5, 10, 20))
    # [10, 10, 10, 20]
\end{lstlisting}
\newpage

%-------------------------------------------------------------------------------
%level=3
\begin{center} Генераторы/Сопроцедуры \end{center}
\begin{itemize}
    \item Генераторы это функции, которые могут сохранить свое состояние, вернуть 
        значение в вызывающую функцию и позже продолжить исполнение с точки остановки
    \item Для приостановки исполнения используется ключевое слово \lstinline!yield!
    \item Функции-генераторы могут использоваться в \LARGE\lstinline!for! циклах и везде, где
        принимается итератор
    \item \lstinline!return! со значением запрещен
\end{itemize}
\vspace{15pt}
\begin{lstlisting}
    def my_generator(value):
        while value >= 0:
            yield value ** 2
            value -= 1

    for val in my_generator(10):
        print val
\end{lstlisting}
\newpage

%-------------------------------------------------------------------------------
%level=3
\begin{center} Генераторы/Сопроцедуры \end{center}
\begin{itemize}
    \item Функция содержащая \lstinline!yield! при вызове возвращает generator
        при этом ни одна строка функции не исполняется, указатель исполнения стоит 
        на первой строке функции
    \item \lstinline!next(generator)! продолжает исполнение до следующего 
        \lstinline!yield! или конца функции
    \item \lstinline!yield! приводит к приостановке исполнения и возврату значения
    \item \lstinline!return! приводит к созданию исключения \lstinline!StopIteration!
        которое автоматически обрабатывается циклом \lstinline!for!
    \item Одновременно может существовать сколько угодно генераторов, созданных из одной функции.
        Все они будут иметь свой набор локальных переменных и указатель исполнения.
\end{itemize}
\newpage

%-------------------------------------------------------------------------------
%level=3
\begin{center} Генераторы/Сопроцедуры \end{center}
\vspace{15pt}
\begin{lstlisting}
    def my_generator(value):
        print "Enter with value =", value
        while value >= 0:
            yield value ** 2
            value -= 1

    gen1 = my_generator(2)
    gen2 = my_generator(200)
    print next(gen1) 
    #Enter with value = 2
    #4
    print next(gen1)
    #1
    print next(gen1)
    #0
    print next(gen1) # error
\end{lstlisting}
\newpage

%-------------------------------------------------------------------------------
%level=3
\begin{center} Генераторы/Сопроцедуры \end{center}
\begin{itemize}
    \item В генератор можно передать значение через \lstinline!gen.send(val)!
        или ошибку через \lstinline!gen.raise(error)! они будут переданны, как 
        результат \lstinline!yield!
\end{itemize}
\vspace{15pt}
\begin{lstlisting}
    def my_generator(value):
        while True:
            res = (yield value)
            value = res ** 2

    gen = my_generator(1)
    print next(gen) # 1
    print gen.send(10) # 100
    print gen.send(10) # 100
\end{lstlisting}
\newpage

%-------------------------------------------------------------------------------
%level=3
\begin{center} Генераторы/Сопроцедуры \end{center}
\begin{itemize}
    \item Бесконечные генераторы
\end{itemize}
\vspace{15pt}
\begin{lstlisting}
    def numbers():
        while True:
            yield val
            val += 1
\end{lstlisting}
\newpage

%-------------------------------------------------------------------------------
%level=3
\begin{center} itertools \end{center}
\newpage

%-------------------------------------------------------------------------------
%level=3
\begin{center} Декораторы \end{center}
\begin{itemize}
    \item Синтаксический сахар для модификаторов фукций
    \item \lstinline!@name! или \lstinline!@name(params)!
\end{itemize}
\vspace{15pt}
\begin{lstlisting}
    @decorator(x, y)
    def func():
        pass

    # equal to
    def func():
        pass

    func = decorator(x, y)(func)
\end{lstlisting}
\newpage

%-------------------------------------------------------------------------------
%level=3
\begin{center} Декораторы \end{center}
\vspace{15pt}
\begin{lstlisting}
    import functools

    def log_params(func):
        @functools.wraps(func)
        def closure(*dt, **mp):
            res = func(*dt, **mp)
            print "{}({} {}) = {}".format(\
                    func.__name__, dt, mp, res)
            return res
        return closure

    @log_params
    def some_func(x, y):
        return x + y

    some_func(1, 2) # some_func((1, 2) {}) = 3
\end{lstlisting}
\newpage

%-------------------------------------------------------------------------------
%level=3
\begin{center} lambda \end{center}
\begin{itemize}
    \item Удобный синтаксис для однострочных функций
    \item \lstinline!lambda! параметры : выражение
    \item Запрещенны print, for, if, ....
\end{itemize}
\vspace{15pt}
\begin{lstlisting}
    map((lambda x : x * 2), (1,2 )) == [2, 4]
    cmp = lambda x, y : x > y
\end{lstlisting}
\newpage

%-------------------------------------------------------------------------------
%level=3
\begin{center} Перегрузка функций \end{center}
\newpage

%-------------------------------------------------------------------------------
\begin{center} AA \end{center}
\begin{itemize}
    \item Написать функцию map следующими способами: рекурсивным (map\_rq), 
        на генераторе (map\_yield) и рекурсивно на генераторе(map\_rq\_yield).
        Функция map принимает два параметра - функцию с одним параметром и итерируемый объект. 
        Возвращает итерируемый объект (список или генератор), котороый состоит из результатов
        применения параметра-функции к элементам итерируемого параметра с сохранением последовательности.
        \lstinline!map(lambda x : x ** 2, [1,2,3]) == [1, 4, 9]!.
        Рекурсивные функции должны обрабатывать не более одного элемента на шаг рекурсии.
        Функции-генераторы должны обрабатывать элементы по мере запроса значений генератора.
\end{itemize}
\newpage

%-------------------------------------------------------------------------------
\begin{center} AA \end{center}
\begin{itemize}
    \item Написать декоратор time\_me, который делает профилирование функции, 
        используя заданную функцию времени. Вторым параметром передается словарь,
        который нужно обновлять с каждым вызовом. В его елемент 'num\_calls' нужно
        заносить количество вызовов, в его элемент 'cum\_time' суммарное время.
\end{itemize}
\vspace{15pt}
\begin{lstlisting}
    import time

    statistic = {}
    @time_me(time.time, statistic)
    def som_func(x, y):
        time.sleep(1.1)

    time_me(1, 2)
    time_me(1, 2)

    assert statistic['num_calls'] == 2
    assert 2.5 > statistic['cum_time'] > 2
\end{lstlisting}
\newpage

%-------------------------------------------------------------------------------
\begin{center} ДЗ 2.1 \end{center}
\begin{itemize}
    \item Написать функцию bind который позволяет связать функцию-параметр с любыми заданными 
        параметрами. По итогу должна получиться функция, которая принимает недостающие параметры
        и вызывает функцию-параметр. Корректность введенных параметров проверять не надо, просто 
        вызвать исходную функцию со слитым набором.
\end{itemize}
\vspace{15pt}
\begin{lstlisting}
    def func(x, y, z, t):
        return x, y, z, t

    f1 = bind(func, 1, 2, t=13)
    assert f1([4]) == (1, 2, [4], 13)
\end{lstlisting}
\newpage

%-------------------------------------------------------------------------------
\begin{center} ДЗ 2.2 \end{center}
\begin{itemize}
    \item Написать декоратор me\_haskell, который позволяет функции вести себя как функция в haskell.
        Она принимает параметры и возвращает новые функции до тех пор, пока накопленных
        за все прошлые вызовы параметров не хватает для вызова. Как только их хватает для 
        вызова - вызывается оригинальная функция и возвращается полученный результат.
        По мере накопления параметров делает проверка их корректности. 
        Запрещается передавать один параметр два раза и передавать параметры, которые не 
        подходят под спецификацию оригинальной функции. При неверных параметрах
        делать \lstinline!raise ValueError(message)!. 
        Для этого посмотреть на функцию \lstinline!inspect.getargspec!.
        Для декорируемой функции должны быть запрещенны * и ** параметры, это
        нужно проверить сразу про декорировании.
\end{itemize}
\vspace{15pt}
\begin{lstlisting}
    @me_haskell
    def func(x, y, z):
        return x, y, z

    f1 = func(1)
    assert f1(2, 3) == (1, 2, 3)
    
    f2 = f1(z = True)
    assert f2("abc") == (1, "abc", True)

    func(1, x=1) # should throw an exception
    func(1, 2)(y=1) # should throw an exception
    func(y=12)(1, 2) # should throw an exception (y defined twice)
    assert func(y=12)(1, z=2) == (1, 12, 2)
\end{lstlisting}
\newpage

%-------------------------------------------------------------------------------
\begin{center} ДЗ 3 \end{center}
\begin{itemize}
    \item Написать декоратор check\_me, который проверяет типа параметров функции.
        Ограничения на типы переменных задаются в докстринг. При неверном
        значении делает \lstinline!raise TypeError(mess)! mess - строка,
        описывающая какой параметр имеет неверный тип и какой тип должен быть.
        Для этого посмотреть на функцию \lstinline!inspect.getargspec!.
        Для декорируемой функции должны быть запрещенны * и ** параметры, это
        нужно проверить сразу про декорировании.
\end{itemize}

\vspace{15pt}
\begin{lstlisting}
    @check_me
    def my_func(x, y):
        """
        @param x: int
        @param y: str
        """
        return x + str(y)

    my_func(1, 2) # TypeError
    my_func(2, "3")# ok
\end{lstlisting}
\newpage

%-------------------------------------------------------------------------------
\end{document}
