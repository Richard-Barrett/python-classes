% python classes slides - exceptions
% (c) 2012 Kostiantyn Danylov aka koder 
% koder.mail@gmail.com
% distributed under CC-BY licence
% http://creativecommons.org/licenses/by/3.0/deed.en

\documentclass{article}
% XeLaTeX
\usepackage{xltxtra}
\usepackage{xunicode}
\usepackage{listings}
\usepackage[landscape]{geometry}

% Fonts
\setmainfont{DejaVu Sans} %{Arial}
\newfontfamily\cyrillicfont{Nimbus Roman No9 L} %{Arial}
\setmonofont{Courier New}
%\setmonofont{Ubuntu Mono}

%\setmonofont{DejaVu Sans Mono}

% Lang
\usepackage{polyglossia}
\setmainlanguage{russian}
\setotherlanguage{english}
\usepackage[dvipsnames,table]{xcolor}


\ifx\pdfoutput\undefined
\usepackage{graphicx}
\else
\usepackage[pdftex]{graphicx}
\fi

\lstset{
	language=python,
	keywordstyle=\color{Emerald},%\texttt, 
	commentstyle=\color{OliveGreen},%\texttt,
	stringstyle=\color{Bittersweet},%\texttt,
	tabsize=4,
	numbers=left,
	xleftmargin=10pt,
	morekeywords={with,as},	
	numberstyle=\large,
	%identifierstyle=\texttt,
	%basicstyle=\texttt,
}

\usepackage{hyperref}

\hypersetup{
	colorlinks=true,
	urlcolor=blue
}

\usepackage{float}
%\floatstyle{boxed} 
%\restylefloat{figure}
\usepackage[normalem]{ulem}


\makeatletter
\def\PY@reset{\let\PY@it=\relax \let\PY@bf=\relax%
    \let\PY@ul=\relax \let\PY@tc=\relax%
    \let\PY@bc=\relax \let\PY@ff=\relax}
\def\PY@tok#1{\csname PY@tok@#1\endcsname}
\def\PY@toks#1+{\ifx\relax#1\empty\else%
    \PY@tok{#1}\expandafter\PY@toks\fi}
\def\PY@do#1{\PY@bc{\PY@tc{\PY@ul{%
    \PY@it{\PY@bf{\PY@ff{#1}}}}}}}
\def\PY#1#2{\PY@reset\PY@toks#1+\relax+\PY@do{#2}}

\expandafter\def\csname PY@tok@gd\endcsname{\def\PY@tc##1{\textcolor[rgb]{0.63,0.00,0.00}{##1}}}
\expandafter\def\csname PY@tok@gu\endcsname{\let\PY@bf=\textbf\def\PY@tc##1{\textcolor[rgb]{0.50,0.00,0.50}{##1}}}
\expandafter\def\csname PY@tok@gt\endcsname{\def\PY@tc##1{\textcolor[rgb]{0.00,0.25,0.82}{##1}}}
\expandafter\def\csname PY@tok@gs\endcsname{\let\PY@bf=\textbf}
\expandafter\def\csname PY@tok@gr\endcsname{\def\PY@tc##1{\textcolor[rgb]{1.00,0.00,0.00}{##1}}}
\expandafter\def\csname PY@tok@cm\endcsname{\let\PY@it=\textit\def\PY@tc##1{\textcolor[rgb]{0.25,0.50,0.50}{##1}}}
\expandafter\def\csname PY@tok@vg\endcsname{\def\PY@tc##1{\textcolor[rgb]{0.10,0.09,0.49}{##1}}}
\expandafter\def\csname PY@tok@m\endcsname{\def\PY@tc##1{\textcolor[rgb]{0.40,0.40,0.40}{##1}}}
\expandafter\def\csname PY@tok@mh\endcsname{\def\PY@tc##1{\textcolor[rgb]{0.40,0.40,0.40}{##1}}}
\expandafter\def\csname PY@tok@go\endcsname{\def\PY@tc##1{\textcolor[rgb]{0.50,0.50,0.50}{##1}}}
\expandafter\def\csname PY@tok@ge\endcsname{\let\PY@it=\textit}
\expandafter\def\csname PY@tok@vc\endcsname{\def\PY@tc##1{\textcolor[rgb]{0.10,0.09,0.49}{##1}}}
\expandafter\def\csname PY@tok@il\endcsname{\def\PY@tc##1{\textcolor[rgb]{0.40,0.40,0.40}{##1}}}
\expandafter\def\csname PY@tok@cs\endcsname{\let\PY@it=\textit\def\PY@tc##1{\textcolor[rgb]{0.25,0.50,0.50}{##1}}}
\expandafter\def\csname PY@tok@cp\endcsname{\def\PY@tc##1{\textcolor[rgb]{0.74,0.48,0.00}{##1}}}
\expandafter\def\csname PY@tok@gi\endcsname{\def\PY@tc##1{\textcolor[rgb]{0.00,0.63,0.00}{##1}}}
\expandafter\def\csname PY@tok@gh\endcsname{\let\PY@bf=\textbf\def\PY@tc##1{\textcolor[rgb]{0.00,0.00,0.50}{##1}}}
\expandafter\def\csname PY@tok@ni\endcsname{\let\PY@bf=\textbf\def\PY@tc##1{\textcolor[rgb]{0.60,0.60,0.60}{##1}}}
\expandafter\def\csname PY@tok@nl\endcsname{\def\PY@tc##1{\textcolor[rgb]{0.63,0.63,0.00}{##1}}}
\expandafter\def\csname PY@tok@nn\endcsname{\let\PY@bf=\textbf\def\PY@tc##1{\textcolor[rgb]{0.00,0.00,1.00}{##1}}}
\expandafter\def\csname PY@tok@no\endcsname{\def\PY@tc##1{\textcolor[rgb]{0.53,0.00,0.00}{##1}}}
\expandafter\def\csname PY@tok@na\endcsname{\def\PY@tc##1{\textcolor[rgb]{0.49,0.56,0.16}{##1}}}
\expandafter\def\csname PY@tok@nb\endcsname{\def\PY@tc##1{\textcolor[rgb]{0.00,0.50,0.00}{##1}}}
\expandafter\def\csname PY@tok@nc\endcsname{\let\PY@bf=\textbf\def\PY@tc##1{\textcolor[rgb]{0.00,0.00,1.00}{##1}}}
\expandafter\def\csname PY@tok@nd\endcsname{\def\PY@tc##1{\textcolor[rgb]{0.67,0.13,1.00}{##1}}}
\expandafter\def\csname PY@tok@ne\endcsname{\let\PY@bf=\textbf\def\PY@tc##1{\textcolor[rgb]{0.82,0.25,0.23}{##1}}}
\expandafter\def\csname PY@tok@nf\endcsname{\def\PY@tc##1{\textcolor[rgb]{0.00,0.00,1.00}{##1}}}
\expandafter\def\csname PY@tok@si\endcsname{\let\PY@bf=\textbf\def\PY@tc##1{\textcolor[rgb]{0.73,0.40,0.53}{##1}}}
\expandafter\def\csname PY@tok@s2\endcsname{\def\PY@tc##1{\textcolor[rgb]{0.73,0.13,0.13}{##1}}}
\expandafter\def\csname PY@tok@vi\endcsname{\def\PY@tc##1{\textcolor[rgb]{0.10,0.09,0.49}{##1}}}
\expandafter\def\csname PY@tok@nt\endcsname{\let\PY@bf=\textbf\def\PY@tc##1{\textcolor[rgb]{0.00,0.50,0.00}{##1}}}
\expandafter\def\csname PY@tok@nv\endcsname{\def\PY@tc##1{\textcolor[rgb]{0.10,0.09,0.49}{##1}}}
\expandafter\def\csname PY@tok@s1\endcsname{\def\PY@tc##1{\textcolor[rgb]{0.73,0.13,0.13}{##1}}}
\expandafter\def\csname PY@tok@sh\endcsname{\def\PY@tc##1{\textcolor[rgb]{0.73,0.13,0.13}{##1}}}
\expandafter\def\csname PY@tok@sc\endcsname{\def\PY@tc##1{\textcolor[rgb]{0.73,0.13,0.13}{##1}}}
\expandafter\def\csname PY@tok@sx\endcsname{\def\PY@tc##1{\textcolor[rgb]{0.00,0.50,0.00}{##1}}}
\expandafter\def\csname PY@tok@bp\endcsname{\def\PY@tc##1{\textcolor[rgb]{0.00,0.50,0.00}{##1}}}
\expandafter\def\csname PY@tok@c1\endcsname{\let\PY@it=\textit\def\PY@tc##1{\textcolor[rgb]{0.25,0.50,0.50}{##1}}}
\expandafter\def\csname PY@tok@kc\endcsname{\let\PY@bf=\textbf\def\PY@tc##1{\textcolor[rgb]{0.00,0.50,0.00}{##1}}}
\expandafter\def\csname PY@tok@c\endcsname{\let\PY@it=\textit\def\PY@tc##1{\textcolor[rgb]{0.25,0.50,0.50}{##1}}}
\expandafter\def\csname PY@tok@mf\endcsname{\def\PY@tc##1{\textcolor[rgb]{0.40,0.40,0.40}{##1}}}
\expandafter\def\csname PY@tok@err\endcsname{\def\PY@bc##1{\setlength{\fboxsep}{0pt}\fcolorbox[rgb]{1.00,0.00,0.00}{1,1,1}{\strut ##1}}}
\expandafter\def\csname PY@tok@kd\endcsname{\let\PY@bf=\textbf\def\PY@tc##1{\textcolor[rgb]{0.00,0.50,0.00}{##1}}}
\expandafter\def\csname PY@tok@ss\endcsname{\def\PY@tc##1{\textcolor[rgb]{0.10,0.09,0.49}{##1}}}
\expandafter\def\csname PY@tok@sr\endcsname{\def\PY@tc##1{\textcolor[rgb]{0.73,0.40,0.53}{##1}}}
\expandafter\def\csname PY@tok@mo\endcsname{\def\PY@tc##1{\textcolor[rgb]{0.40,0.40,0.40}{##1}}}
\expandafter\def\csname PY@tok@kn\endcsname{\let\PY@bf=\textbf\def\PY@tc##1{\textcolor[rgb]{0.00,0.50,0.00}{##1}}}
\expandafter\def\csname PY@tok@mi\endcsname{\def\PY@tc##1{\textcolor[rgb]{0.40,0.40,0.40}{##1}}}
\expandafter\def\csname PY@tok@gp\endcsname{\let\PY@bf=\textbf\def\PY@tc##1{\textcolor[rgb]{0.00,0.00,0.50}{##1}}}
\expandafter\def\csname PY@tok@o\endcsname{\def\PY@tc##1{\textcolor[rgb]{0.40,0.40,0.40}{##1}}}
\expandafter\def\csname PY@tok@kr\endcsname{\let\PY@bf=\textbf\def\PY@tc##1{\textcolor[rgb]{0.00,0.50,0.00}{##1}}}
\expandafter\def\csname PY@tok@s\endcsname{\def\PY@tc##1{\textcolor[rgb]{0.73,0.13,0.13}{##1}}}
\expandafter\def\csname PY@tok@kp\endcsname{\def\PY@tc##1{\textcolor[rgb]{0.00,0.50,0.00}{##1}}}
\expandafter\def\csname PY@tok@w\endcsname{\def\PY@tc##1{\textcolor[rgb]{0.73,0.73,0.73}{##1}}}
\expandafter\def\csname PY@tok@kt\endcsname{\def\PY@tc##1{\textcolor[rgb]{0.69,0.00,0.25}{##1}}}
\expandafter\def\csname PY@tok@ow\endcsname{\let\PY@bf=\textbf\def\PY@tc##1{\textcolor[rgb]{0.67,0.13,1.00}{##1}}}
\expandafter\def\csname PY@tok@sb\endcsname{\def\PY@tc##1{\textcolor[rgb]{0.73,0.13,0.13}{##1}}}
\expandafter\def\csname PY@tok@k\endcsname{\let\PY@bf=\textbf\def\PY@tc##1{\textcolor[rgb]{0.00,0.50,0.00}{##1}}}
\expandafter\def\csname PY@tok@se\endcsname{\let\PY@bf=\textbf\def\PY@tc##1{\textcolor[rgb]{0.73,0.40,0.13}{##1}}}
\expandafter\def\csname PY@tok@sd\endcsname{\let\PY@it=\textit\def\PY@tc##1{\textcolor[rgb]{0.73,0.13,0.13}{##1}}}

\def\PYZbs{\char`\\}
\def\PYZus{\char`\_}
\def\PYZob{\char`\{}
\def\PYZcb{\char`\}}
\def\PYZca{\char`\^}
\def\PYZam{\char`\&}
\def\PYZlt{\char`\<}
\def\PYZgt{\char`\>}
\def\PYZsh{\char`\#}
\def\PYZpc{\char`\%}
\def\PYZdl{\char`\$}
\def\PYZti{\char`\~}
% for compatibility with earlier versions
\def\PYZat{@}
\def\PYZlb{[}
\def\PYZrb{]}
\makeatother


\begin{document}
\LARGE

%-------------------------------------------------------------------------------
\begin{center} Как написать надежную обработку ошибок? \end{center}
70е:
\begin{itemize}
	\item Для каждой функции выделить специальное значение - признак ошибки
	\item Проверять результат каждой функции и в случае ошибки обрабатывать
	        ее или передавать дальше, если обработка в текущей точке невозможна.
\end{itemize}
\begin{lstlisting}
	def do_some_work(name, vals):
		if not isinstance(name, basestring):
			return None
		#......
		return res

	def f2():
		res = do_some_work("1231", [1, 2])
		if res is None:
			return None
		.....
\end{lstlisting}
\newpage

%-------------------------------------------------------------------------------
\begin{center} Проблема 1 \end{center}
\begin{itemize}
	\item Нужно помнить какое значение возвращает эта функция при ошибке.
	\item Нужно хранить дополнительно информацию о ошибке
	\item Что-бы информация о ошибке имела смысл - присоединить к ошибке контекст
\end{itemize}
\newpage

%-------------------------------------------------------------------------------
\begin{center} Решение \end{center}
\begin{itemize}
	\item Возвращать из каждой функции тройку (is\_ok, stack, result)
	\item Превратить return res в return (True, None, res)
	\item Превратить return err в return (False, [curr\_func\_name], err_description)
	\item В обработчиках ошибок нужно фильтровать ошибки по типу
	\item Проверять на выходе из каждой функции результат
	\item Если не проверить ошибку, то проблема 
			возникнет в непредсказуемом месте кода
\end{itemize}
\newpage

%-------------------------------------------------------------------------------
\begin{lstlisting}
	def do_some_work(name, vals):
		if not isinstance(name, basestring):
			return (False, ["do_some_work"], "name should be a string")
		#......
		return (True, None, res)

	def f2():
		is_ok, stack, res = do_some_work("1231", [1, 2])
		if not is_ok:
			return (False, stack + ["f2"], res)
		.....
\end{lstlisting}
\newpage

%-------------------------------------------------------------------------------
\begin{center} Еще проблемы \end{center}
\begin{itemize}
	\item Загрязняет код
	\item Для каждого вызова отдельная строка и свой if
	\item Такие ошибки часто не проверяются (printf)
	\item
	\item Вспомагательный код очень простой - его 
		  генерацию можно переложить на компилятор
\end{itemize}
\newpage

%-------------------------------------------------------------------------------
\begin{center} Именно это и делают современные языки \end{center}
\begin{center} Исключения \end{center}
\begin{figure}[ht]
\begin{minipage}[b]{0.45\linewidth}
\Large
\begin{lstlisting}
	def x1(a, b):
		if 0 == b:
			return (False, 
				    ["x1"],
				    ZeroDiv)

	def x2(a, b):
		print a,b
		ok, res, stack = x1(a, b)
		if not ok:
			return (False, 
					stack + \
					["x2"],
					res)

	ok, stack, res = x1(a, b)
	if not ok and res is ZeroDiv:
		#process
\end{lstlisting}
\end{minipage}
\hspace{2cm}
\begin{minipage}[b]{0.45\linewidth}
\Large
\begin{lstlisting}
	def x1(a, b):
		if 0 == b:
			raise ZeroDivisionError()



	def x2(a, b):
		print a,b
		return x1(a, b)




	try:
		res = x1(a, b)
	except ZeroDivisionError:
		#process
\end{lstlisting}
\end{minipage}
\end{figure}
\newpage

%-------------------------------------------------------------------------------
\begin{center} Исключения \end{center}
\begin{itemize}
	\item Исключение – это событие, после которого дальнейшее продолжение 
			работы в данной точке бессмысленно. По итогу такого события 
			генерируется объект-исключение, и исполнение передается обработчику 
			ошибок этого типа
	\item Пример – деление на 0, выбрасывается ошибка ZeroDivisionError
	\item Исключения помогают упростить код, убрав из него множество 
			проверок и значительно облегчить восстановление программы после сбоя
	\item Исключения упрощают доставку информации о ошибке от той точке, 
			в которой она возникла к той точке где она может быть обработанны
	\item Типы всех исключений дожны наследовать Exception
	\item Чаще всего принимают строку как параметр
\end{itemize}
\newpage

%-------------------------------------------------------------------------------

\begin{center} Исключения \end{center}
\begin{lstlisting}
	try:
		block1
	except tp2 as var2:
		block2
	except (tp3, tp4) as var3:
		block3
	else:
		block5
	finally:
		block4
\end{lstlisting}
\newpage

%-------------------------------------------------------------------------------
\begin{center} Исключения \end{center}
\begin{lstlisting}
	try:
		raise tp2("xxx")  # <<<<
	except tp2 as var2:
		block2            # <<<<
	except (tp3, tp4) as var3:
		block3
	else:
		block5
	finally:
		block4            # <<<<
\end{lstlisting}
\newpage

%-------------------------------------------------------------------------------
\begin{center} Исключения \end{center}
\begin{lstlisting}
	try:
		pass           # <<<
	except tp2 as var2:
		block2
	except (tp3, tp4) as var3:
		block3
	else:
		block5         # <<<
	finally:
		block4         # <<<
\end{lstlisting}
\newpage

%-------------------------------------------------------------------------------
\begin{center} Исключения \end{center}
\begin{lstlisting}
	def f1(t, d, x, y):
		if t – d  == 0:
		    return None
		else:
		    t1 = ((x + y) / (t - d))
		    if t1 == 0:
		        return None
		    else:
		        return 1 / ((x + y) / (t - d))

	def f2(t, d, x, y):
		try:
		    return 1 / ((x + y) / (t - d))
		except ZeroDivisionError:
		    return None
\end{lstlisting}
\newpage

%-------------------------------------------------------------------------------
\begin{center} Исключения. raise \end{center}
\begin{itemize}
	\item \lstinline!raise ExceptionType(....)! порождает исключение
	\item \lstinline!ExceptionType! должно наследовать \lstinline!Exception!
	\item \lstinline!raise! без параметров разрешено только в блоке except. 
		При этом повторно выбрасывается текущее исключение
\end{itemize}
\begin{lstlisting}
	try:
		func()
	except Exception:
		print "func cause exception"
		raise
\end{lstlisting}
\newpage

%-------------------------------------------------------------------------------
\begin{center} Стандартные исключения \end{center}
\newpage

%-------------------------------------------------------------------------------
\begin{center} Исключения. traceback \end{center}
	В обработчике исключения \lstinline!sys.exc_info()! возвращает тройку
		(Тип исключения, Объект исключения, Состояние Стека)
\begin{lstlisting}
	try:
	    raise ValueError("ddd")
	except Exception as x:
		tb = sys.exc_info()[2]

	print tb.tb_frame # <frame at 0x....>
	print tb.tb_frame.f_lineno # 4
	print tb.tb_frame.f_code.co_name # '<module>'
	print tb.tb_frame.f_code.co_filename 
		# '<ipython-input-7-492d537cf800>'
	print tb.tb_next # <frame at 0x....> or None
	del tb
\end{lstlisting}
\newpage

%-------------------------------------------------------------------------------
\begin{center} Гарантии безопасности исключений \end{center}
\begin{lstlisting}
	class TimedCallbackStack(object):
		def __init__(self, tf):
			self.cb_list = []
			self.tf = tf
			self.sum_time = 0.0

		def pop_and_exec(self):
			func = self.cb_list.pop()
			t = self.tf()
			res = func()
			self.sum_time += self.tf() - t
			return res
\end{lstlisting}

%-------------------------------------------------------------------------------
\begin{center} Другие проблемы исключений \end{center}
\begin{itemize}
	\item Выполнение функции может прерваться в любой точке - нужно работать со всеми ресурсами
	      через try/finally или, лучше через with.
	\item try/finally/with делают из линейного кода вложенный
	\item Для не локальных объектов все печально
	\item Все равно нужно помнить какие исключения порождает конкретная функция
	      (хотя все не так плохо, как с "исключениями" из 70х)
\end{itemize}
\newpage

%-------------------------------------------------------------------------------
\end{document}

