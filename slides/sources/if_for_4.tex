% python classes slides - control statements
% (c) 2012 Kostiantyn Danylov aka koder 
% koder.mail@gmail.com
% distributed under CC-BY licence
% http://creativecommons.org/licenses/by/3.0/deed.en

\documentclass{article}
% XeLaTeX
\usepackage{xltxtra}
\usepackage{xunicode}
\usepackage{listings}
\usepackage[landscape]{geometry}

% Fonts
\setmainfont{DejaVu Sans} %{Arial}
\newfontfamily\cyrillicfont{Nimbus Roman No9 L} %{Arial}
\setmonofont{Courier New}
%\setmonofont{Ubuntu Mono}

%\setmonofont{DejaVu Sans Mono}

% Lang
\usepackage{polyglossia}
\setmainlanguage{russian}
\setotherlanguage{english}
\usepackage[dvipsnames,table]{xcolor}


\ifx\pdfoutput\undefined
\usepackage{graphicx}
\else
\usepackage[pdftex]{graphicx}
\fi

\lstset{
	language=python,
	keywordstyle=\color{Emerald},%\texttt, 
	commentstyle=\color{OliveGreen},%\texttt,
	stringstyle=\color{Bittersweet},%\texttt,
	tabsize=4,
	numbers=left,
	xleftmargin=10pt,
	morekeywords={with,as},	
	numberstyle=\large,
	%identifierstyle=\texttt,
	%basicstyle=\texttt,
}

\usepackage{hyperref}

\hypersetup{
	colorlinks=true,
	urlcolor=blue
}

\usepackage{float}
%\floatstyle{boxed} 
%\restylefloat{figure}
\usepackage[normalem]{ulem}


\makeatletter
\def\PY@reset{\let\PY@it=\relax \let\PY@bf=\relax%
    \let\PY@ul=\relax \let\PY@tc=\relax%
    \let\PY@bc=\relax \let\PY@ff=\relax}
\def\PY@tok#1{\csname PY@tok@#1\endcsname}
\def\PY@toks#1+{\ifx\relax#1\empty\else%
    \PY@tok{#1}\expandafter\PY@toks\fi}
\def\PY@do#1{\PY@bc{\PY@tc{\PY@ul{%
    \PY@it{\PY@bf{\PY@ff{#1}}}}}}}
\def\PY#1#2{\PY@reset\PY@toks#1+\relax+\PY@do{#2}}

\expandafter\def\csname PY@tok@gd\endcsname{\def\PY@tc##1{\textcolor[rgb]{0.63,0.00,0.00}{##1}}}
\expandafter\def\csname PY@tok@gu\endcsname{\let\PY@bf=\textbf\def\PY@tc##1{\textcolor[rgb]{0.50,0.00,0.50}{##1}}}
\expandafter\def\csname PY@tok@gt\endcsname{\def\PY@tc##1{\textcolor[rgb]{0.00,0.25,0.82}{##1}}}
\expandafter\def\csname PY@tok@gs\endcsname{\let\PY@bf=\textbf}
\expandafter\def\csname PY@tok@gr\endcsname{\def\PY@tc##1{\textcolor[rgb]{1.00,0.00,0.00}{##1}}}
\expandafter\def\csname PY@tok@cm\endcsname{\let\PY@it=\textit\def\PY@tc##1{\textcolor[rgb]{0.25,0.50,0.50}{##1}}}
\expandafter\def\csname PY@tok@vg\endcsname{\def\PY@tc##1{\textcolor[rgb]{0.10,0.09,0.49}{##1}}}
\expandafter\def\csname PY@tok@m\endcsname{\def\PY@tc##1{\textcolor[rgb]{0.40,0.40,0.40}{##1}}}
\expandafter\def\csname PY@tok@mh\endcsname{\def\PY@tc##1{\textcolor[rgb]{0.40,0.40,0.40}{##1}}}
\expandafter\def\csname PY@tok@go\endcsname{\def\PY@tc##1{\textcolor[rgb]{0.50,0.50,0.50}{##1}}}
\expandafter\def\csname PY@tok@ge\endcsname{\let\PY@it=\textit}
\expandafter\def\csname PY@tok@vc\endcsname{\def\PY@tc##1{\textcolor[rgb]{0.10,0.09,0.49}{##1}}}
\expandafter\def\csname PY@tok@il\endcsname{\def\PY@tc##1{\textcolor[rgb]{0.40,0.40,0.40}{##1}}}
\expandafter\def\csname PY@tok@cs\endcsname{\let\PY@it=\textit\def\PY@tc##1{\textcolor[rgb]{0.25,0.50,0.50}{##1}}}
\expandafter\def\csname PY@tok@cp\endcsname{\def\PY@tc##1{\textcolor[rgb]{0.74,0.48,0.00}{##1}}}
\expandafter\def\csname PY@tok@gi\endcsname{\def\PY@tc##1{\textcolor[rgb]{0.00,0.63,0.00}{##1}}}
\expandafter\def\csname PY@tok@gh\endcsname{\let\PY@bf=\textbf\def\PY@tc##1{\textcolor[rgb]{0.00,0.00,0.50}{##1}}}
\expandafter\def\csname PY@tok@ni\endcsname{\let\PY@bf=\textbf\def\PY@tc##1{\textcolor[rgb]{0.60,0.60,0.60}{##1}}}
\expandafter\def\csname PY@tok@nl\endcsname{\def\PY@tc##1{\textcolor[rgb]{0.63,0.63,0.00}{##1}}}
\expandafter\def\csname PY@tok@nn\endcsname{\let\PY@bf=\textbf\def\PY@tc##1{\textcolor[rgb]{0.00,0.00,1.00}{##1}}}
\expandafter\def\csname PY@tok@no\endcsname{\def\PY@tc##1{\textcolor[rgb]{0.53,0.00,0.00}{##1}}}
\expandafter\def\csname PY@tok@na\endcsname{\def\PY@tc##1{\textcolor[rgb]{0.49,0.56,0.16}{##1}}}
\expandafter\def\csname PY@tok@nb\endcsname{\def\PY@tc##1{\textcolor[rgb]{0.00,0.50,0.00}{##1}}}
\expandafter\def\csname PY@tok@nc\endcsname{\let\PY@bf=\textbf\def\PY@tc##1{\textcolor[rgb]{0.00,0.00,1.00}{##1}}}
\expandafter\def\csname PY@tok@nd\endcsname{\def\PY@tc##1{\textcolor[rgb]{0.67,0.13,1.00}{##1}}}
\expandafter\def\csname PY@tok@ne\endcsname{\let\PY@bf=\textbf\def\PY@tc##1{\textcolor[rgb]{0.82,0.25,0.23}{##1}}}
\expandafter\def\csname PY@tok@nf\endcsname{\def\PY@tc##1{\textcolor[rgb]{0.00,0.00,1.00}{##1}}}
\expandafter\def\csname PY@tok@si\endcsname{\let\PY@bf=\textbf\def\PY@tc##1{\textcolor[rgb]{0.73,0.40,0.53}{##1}}}
\expandafter\def\csname PY@tok@s2\endcsname{\def\PY@tc##1{\textcolor[rgb]{0.73,0.13,0.13}{##1}}}
\expandafter\def\csname PY@tok@vi\endcsname{\def\PY@tc##1{\textcolor[rgb]{0.10,0.09,0.49}{##1}}}
\expandafter\def\csname PY@tok@nt\endcsname{\let\PY@bf=\textbf\def\PY@tc##1{\textcolor[rgb]{0.00,0.50,0.00}{##1}}}
\expandafter\def\csname PY@tok@nv\endcsname{\def\PY@tc##1{\textcolor[rgb]{0.10,0.09,0.49}{##1}}}
\expandafter\def\csname PY@tok@s1\endcsname{\def\PY@tc##1{\textcolor[rgb]{0.73,0.13,0.13}{##1}}}
\expandafter\def\csname PY@tok@sh\endcsname{\def\PY@tc##1{\textcolor[rgb]{0.73,0.13,0.13}{##1}}}
\expandafter\def\csname PY@tok@sc\endcsname{\def\PY@tc##1{\textcolor[rgb]{0.73,0.13,0.13}{##1}}}
\expandafter\def\csname PY@tok@sx\endcsname{\def\PY@tc##1{\textcolor[rgb]{0.00,0.50,0.00}{##1}}}
\expandafter\def\csname PY@tok@bp\endcsname{\def\PY@tc##1{\textcolor[rgb]{0.00,0.50,0.00}{##1}}}
\expandafter\def\csname PY@tok@c1\endcsname{\let\PY@it=\textit\def\PY@tc##1{\textcolor[rgb]{0.25,0.50,0.50}{##1}}}
\expandafter\def\csname PY@tok@kc\endcsname{\let\PY@bf=\textbf\def\PY@tc##1{\textcolor[rgb]{0.00,0.50,0.00}{##1}}}
\expandafter\def\csname PY@tok@c\endcsname{\let\PY@it=\textit\def\PY@tc##1{\textcolor[rgb]{0.25,0.50,0.50}{##1}}}
\expandafter\def\csname PY@tok@mf\endcsname{\def\PY@tc##1{\textcolor[rgb]{0.40,0.40,0.40}{##1}}}
\expandafter\def\csname PY@tok@err\endcsname{\def\PY@bc##1{\setlength{\fboxsep}{0pt}\fcolorbox[rgb]{1.00,0.00,0.00}{1,1,1}{\strut ##1}}}
\expandafter\def\csname PY@tok@kd\endcsname{\let\PY@bf=\textbf\def\PY@tc##1{\textcolor[rgb]{0.00,0.50,0.00}{##1}}}
\expandafter\def\csname PY@tok@ss\endcsname{\def\PY@tc##1{\textcolor[rgb]{0.10,0.09,0.49}{##1}}}
\expandafter\def\csname PY@tok@sr\endcsname{\def\PY@tc##1{\textcolor[rgb]{0.73,0.40,0.53}{##1}}}
\expandafter\def\csname PY@tok@mo\endcsname{\def\PY@tc##1{\textcolor[rgb]{0.40,0.40,0.40}{##1}}}
\expandafter\def\csname PY@tok@kn\endcsname{\let\PY@bf=\textbf\def\PY@tc##1{\textcolor[rgb]{0.00,0.50,0.00}{##1}}}
\expandafter\def\csname PY@tok@mi\endcsname{\def\PY@tc##1{\textcolor[rgb]{0.40,0.40,0.40}{##1}}}
\expandafter\def\csname PY@tok@gp\endcsname{\let\PY@bf=\textbf\def\PY@tc##1{\textcolor[rgb]{0.00,0.00,0.50}{##1}}}
\expandafter\def\csname PY@tok@o\endcsname{\def\PY@tc##1{\textcolor[rgb]{0.40,0.40,0.40}{##1}}}
\expandafter\def\csname PY@tok@kr\endcsname{\let\PY@bf=\textbf\def\PY@tc##1{\textcolor[rgb]{0.00,0.50,0.00}{##1}}}
\expandafter\def\csname PY@tok@s\endcsname{\def\PY@tc##1{\textcolor[rgb]{0.73,0.13,0.13}{##1}}}
\expandafter\def\csname PY@tok@kp\endcsname{\def\PY@tc##1{\textcolor[rgb]{0.00,0.50,0.00}{##1}}}
\expandafter\def\csname PY@tok@w\endcsname{\def\PY@tc##1{\textcolor[rgb]{0.73,0.73,0.73}{##1}}}
\expandafter\def\csname PY@tok@kt\endcsname{\def\PY@tc##1{\textcolor[rgb]{0.69,0.00,0.25}{##1}}}
\expandafter\def\csname PY@tok@ow\endcsname{\let\PY@bf=\textbf\def\PY@tc##1{\textcolor[rgb]{0.67,0.13,1.00}{##1}}}
\expandafter\def\csname PY@tok@sb\endcsname{\def\PY@tc##1{\textcolor[rgb]{0.73,0.13,0.13}{##1}}}
\expandafter\def\csname PY@tok@k\endcsname{\let\PY@bf=\textbf\def\PY@tc##1{\textcolor[rgb]{0.00,0.50,0.00}{##1}}}
\expandafter\def\csname PY@tok@se\endcsname{\let\PY@bf=\textbf\def\PY@tc##1{\textcolor[rgb]{0.73,0.40,0.13}{##1}}}
\expandafter\def\csname PY@tok@sd\endcsname{\let\PY@it=\textit\def\PY@tc##1{\textcolor[rgb]{0.73,0.13,0.13}{##1}}}

\def\PYZbs{\char`\\}
\def\PYZus{\char`\_}
\def\PYZob{\char`\{}
\def\PYZcb{\char`\}}
\def\PYZca{\char`\^}
\def\PYZam{\char`\&}
\def\PYZlt{\char`\<}
\def\PYZgt{\char`\>}
\def\PYZsh{\char`\#}
\def\PYZpc{\char`\%}
\def\PYZdl{\char`\$}
\def\PYZti{\char`\~}
% for compatibility with earlier versions
\def\PYZat{@}
\def\PYZlb{[}
\def\PYZrb{]}
\makeatother

\begin{document}
\LARGE

%-------------------------------------------------------------------------------
\center{Блоки кода}
\begin{itemize}
	\item Блоки ограничивают участок кода, принадлежащий управляющей конструкции
	\item Начинаются с “:”, которым оканчивается конструкция 
	\item Все строки блока имеют уровень отступа равным начальной строке блока
	\item Отступы делаются с помошью табуляции или пробелов
	\item Блоки могут содержать другие блоки (с более глубокими отступами)
\end{itemize}
\vspace{15pt}
\begin{lstlisting}
	Some_contruction:
		y = 2
		z = x + y
	#end_of_block
\end{lstlisting}
\newpage

%-------------------------------------------------------------------------------
\center{Блоки кода}
\begin{itemize}
	\item Блоки это не области видимости переменных. Переменные видны и после выхода из блока
	\item \lstinline$pass$ – пустой блок
\end{itemize}
\newpage

%-------------------------------------------------------------------------------
\center{if - Условное выполнение участков кода}
\vspace{15pt}
\begin{lstlisting}
	if  condition1 :
	    pass # excuted if condition1 is true
	elif condition2 :
	    pass # excuted if condition1 is false and condition2 is true
	#... 
	else:
	    pass # executed if all conditions is false 
\end{lstlisting}
\newpage

%-------------------------------------------------------------------------------
\center{if}
\vspace{15pt}
\begin{lstlisting}
	x = 12
	sign = 0
	if x > 0:
	    print x, "positive"
	    sign = 1
	elif x < 0:
	    print x, "negative"
	    sign = -1
	else:
	    print x, "== 0"
	    sign = 0
\end{lstlisting}
\newpage

%-------------------------------------------------------------------------------
\center{inline if}
\vspace{15pt}
\begin{lstlisting}
	res = x if x >= 0 else -x
	# res = (x >= 0 ? x : -x)
\end{lstlisting}
\newpage

%-------------------------------------------------------------------------------
\center{while}
\vspace{15pt}
\begin{lstlisting}
	while condition:
		pass # executed while condition is true
	else:
		pass # if no error or break in body

	x = 1
	while x < 100:
		print x, "less than 100"
		x *= 2
\end{lstlisting}
\newpage

%-------------------------------------------------------------------------------
\center{for - цикл по множеству}
\vspace{15pt}
\begin{lstlisting}
	for x in iterable:
		func(x) # for each element in iterable
	else:
		pass # if no error or break in body

	sum = 0
	for x in range(100):
		sum += x
	print x  # 99 * 100 / 2

	for i in range(n): # xrange(n)
	    pass

	n = 121213

    dividers = []
    while n > 3:
        for divider in range(2, int(n ** 0.5) + 1):
            if n % divider == 0:
                break
        else:
            break
        n //= divider
        dividers.append(divider)

    if n != 1:
    	dividers.append(n)
\end{lstlisting}
\newpage

%-------------------------------------------------------------------------------
\center{for undercover}
\vspace{15pt}
\begin{lstlisting}
	for x in container:
	    f(x)

	# some times equal to

	_tmp = 0
	while _tmp < len(container):
	    x = container[_tmp]
	    f(x)
	    _tmp += 1
\end{lstlisting}
\newpage

%-------------------------------------------------------------------------------
\center{break \& сontinue как всегда}
\begin{itemize}
	\item \lstinline!break!  выходит из цикла
	\item \lstinline!continue! переходит к следующей итерации
\end{itemize}
\newpage

%-------------------------------------------------------------------------------
\center{Нет}
\begin{itemize}
	\item goto 
	\item switch + case 
	\item until 
	\item do{}while, do{}until
\end{itemize}
\newpage

%-------------------------------------------------------------------------------
\center{with}
\vspace{15pt}
\begin{lstlisting}
	with expression as var:
		block

	# mostly the same as
	
	var = expression
	var.__enter__()
	
	block
	
	if error_happened:
		if var.__exit__(error_data):
			# pass_error_further
		else:
			# supress_error
	else:
		var.__exit__()
\end{lstlisting}
\newpage

%-------------------------------------------------------------------------------
\center{использование with}
\vspace{15pt}
\begin{lstlisting}
	with open(r“C:\xxx.bin”, "w") as fd:
	    fd.write(“-” * 100 + "\n")
	    fd.write(“+” * 100 + "\n")

	with open(r“C:\xxx.bin”, "r") as fd:
	    for line in fd:
	    	print line

	with db.cursor() as cur:
	    curr.execute(update_request_1)
	    curr.execute(update_request_2)
		# commit or rollback
\end{lstlisting}
\newpage

%-------------------------------------------------------------------------------
\center{List comprehension}
\vspace{15pt}
\begin{lstlisting}
	res = [func(i) for i in some_iter if func2(i)]

	res = ["{:.2f}".format(i ** 0.5) 
				for i in [-1, 0, 1, 2, 3] 
					if i >= 0]
	
	res == ['0.00', '1.00', '1.41', '1.73']

	res = [(i + 0j) ** 0.5 for i in [-1, 0 ,1, 2, 3]]
	res = {func(i) for i in some_iter if func2(i)}
\end{lstlisting}
\newpage

%-------------------------------------------------------------------------------
\center{Функции - минимум}
\begin{lstlisting}
	def func_name1(param1, param2):
		"documentation"
		# block
		x = param1 + param2
		return x

	def func_name2(param1, param2):
		"documentation"
		# block
		x = param1 + param2
		if x > 0:
			return x
		else:
			return 0
\end{lstlisting}

%-------------------------------------------------------------------------------
\center{Unit tests - find}
\begin{lstlisting}
	assert find("abc", "b") == 1
	assert find("abc", "b") == "abc".find("b")

	assert func("abc", "a") == 0
	assert func("abca", "a") == 0
	assert func("dabca", "a") == 1
	assert func("", "a") == -1
	assert func("a", "a") == 0
	assert func("ab", "abc") == 0
	assert func("b" * 1000 + "abc", "abc") == 1000
	assert func("b" * 1000 + "abc", "abcd") == -1

	all_symbols = "".join([chr(i) for i in range(255)])
	assert func(all_symbols, chr(100)) == 100

	assert find("", "") == 0
	assert find("", "") == "".find("")
\end{lstlisting}

%-------------------------------------------------------------------------------
\center{AA}
\begin{itemize}
	\item Написать строковые функции find, replace, 
			split, join использую срезы строк 
			(без применения других методов строк)
	\item Написать интерпретатор подмножества языка forth. 
		  Со следующими командами push, pop, add, print, sub, dup.
\end{itemize}
\newpage

%-------------------------------------------------------------------------------
\end{document}
