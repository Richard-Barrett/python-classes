% python classes slides - refactored lection for khpu
% (c) 2012 Kostiantyn Danylov aka koder 
% koder.mail@gmail.com
% distributed under CC-BY licence
% http://creativecommons.org/licenses/by/3.0/deed.en

\documentclass{article}
% XeLaTeX
\usepackage{xltxtra}
\usepackage{xunicode}
\usepackage{listings}
\usepackage[landscape]{geometry}

% Fonts
\setmainfont{DejaVu Sans} %{Arial}
\newfontfamily\cyrillicfont{Nimbus Roman No9 L} %{Arial}
\setmonofont{Courier New}
%\setmonofont{Ubuntu Mono}

%\setmonofont{DejaVu Sans Mono}

% Lang
\usepackage{polyglossia}
\setmainlanguage{russian}
\setotherlanguage{english}
\usepackage[dvipsnames,table]{xcolor}


\ifx\pdfoutput\undefined
\usepackage{graphicx}
\else
\usepackage[pdftex]{graphicx}
\fi

\lstset{
	language=python,
	keywordstyle=\color{Emerald},%\texttt, 
	commentstyle=\color{OliveGreen},%\texttt,
	stringstyle=\color{Bittersweet},%\texttt,
	tabsize=4,
	numbers=left,
	xleftmargin=10pt,
	morekeywords={with,as},	
	numberstyle=\large,
	%identifierstyle=\texttt,
	%basicstyle=\texttt,
}

\usepackage{hyperref}

\hypersetup{
	colorlinks=true,
	urlcolor=blue
}

\usepackage{float}
%\floatstyle{boxed} 
%\restylefloat{figure}
\usepackage[normalem]{ulem}


\makeatletter
\def\PY@reset{\let\PY@it=\relax \let\PY@bf=\relax%
    \let\PY@ul=\relax \let\PY@tc=\relax%
    \let\PY@bc=\relax \let\PY@ff=\relax}
\def\PY@tok#1{\csname PY@tok@#1\endcsname}
\def\PY@toks#1+{\ifx\relax#1\empty\else%
    \PY@tok{#1}\expandafter\PY@toks\fi}
\def\PY@do#1{\PY@bc{\PY@tc{\PY@ul{%
    \PY@it{\PY@bf{\PY@ff{#1}}}}}}}
\def\PY#1#2{\PY@reset\PY@toks#1+\relax+\PY@do{#2}}

\expandafter\def\csname PY@tok@gd\endcsname{\def\PY@tc##1{\textcolor[rgb]{0.63,0.00,0.00}{##1}}}
\expandafter\def\csname PY@tok@gu\endcsname{\let\PY@bf=\textbf\def\PY@tc##1{\textcolor[rgb]{0.50,0.00,0.50}{##1}}}
\expandafter\def\csname PY@tok@gt\endcsname{\def\PY@tc##1{\textcolor[rgb]{0.00,0.25,0.82}{##1}}}
\expandafter\def\csname PY@tok@gs\endcsname{\let\PY@bf=\textbf}
\expandafter\def\csname PY@tok@gr\endcsname{\def\PY@tc##1{\textcolor[rgb]{1.00,0.00,0.00}{##1}}}
\expandafter\def\csname PY@tok@cm\endcsname{\let\PY@it=\textit\def\PY@tc##1{\textcolor[rgb]{0.25,0.50,0.50}{##1}}}
\expandafter\def\csname PY@tok@vg\endcsname{\def\PY@tc##1{\textcolor[rgb]{0.10,0.09,0.49}{##1}}}
\expandafter\def\csname PY@tok@m\endcsname{\def\PY@tc##1{\textcolor[rgb]{0.40,0.40,0.40}{##1}}}
\expandafter\def\csname PY@tok@mh\endcsname{\def\PY@tc##1{\textcolor[rgb]{0.40,0.40,0.40}{##1}}}
\expandafter\def\csname PY@tok@go\endcsname{\def\PY@tc##1{\textcolor[rgb]{0.50,0.50,0.50}{##1}}}
\expandafter\def\csname PY@tok@ge\endcsname{\let\PY@it=\textit}
\expandafter\def\csname PY@tok@vc\endcsname{\def\PY@tc##1{\textcolor[rgb]{0.10,0.09,0.49}{##1}}}
\expandafter\def\csname PY@tok@il\endcsname{\def\PY@tc##1{\textcolor[rgb]{0.40,0.40,0.40}{##1}}}
\expandafter\def\csname PY@tok@cs\endcsname{\let\PY@it=\textit\def\PY@tc##1{\textcolor[rgb]{0.25,0.50,0.50}{##1}}}
\expandafter\def\csname PY@tok@cp\endcsname{\def\PY@tc##1{\textcolor[rgb]{0.74,0.48,0.00}{##1}}}
\expandafter\def\csname PY@tok@gi\endcsname{\def\PY@tc##1{\textcolor[rgb]{0.00,0.63,0.00}{##1}}}
\expandafter\def\csname PY@tok@gh\endcsname{\let\PY@bf=\textbf\def\PY@tc##1{\textcolor[rgb]{0.00,0.00,0.50}{##1}}}
\expandafter\def\csname PY@tok@ni\endcsname{\let\PY@bf=\textbf\def\PY@tc##1{\textcolor[rgb]{0.60,0.60,0.60}{##1}}}
\expandafter\def\csname PY@tok@nl\endcsname{\def\PY@tc##1{\textcolor[rgb]{0.63,0.63,0.00}{##1}}}
\expandafter\def\csname PY@tok@nn\endcsname{\let\PY@bf=\textbf\def\PY@tc##1{\textcolor[rgb]{0.00,0.00,1.00}{##1}}}
\expandafter\def\csname PY@tok@no\endcsname{\def\PY@tc##1{\textcolor[rgb]{0.53,0.00,0.00}{##1}}}
\expandafter\def\csname PY@tok@na\endcsname{\def\PY@tc##1{\textcolor[rgb]{0.49,0.56,0.16}{##1}}}
\expandafter\def\csname PY@tok@nb\endcsname{\def\PY@tc##1{\textcolor[rgb]{0.00,0.50,0.00}{##1}}}
\expandafter\def\csname PY@tok@nc\endcsname{\let\PY@bf=\textbf\def\PY@tc##1{\textcolor[rgb]{0.00,0.00,1.00}{##1}}}
\expandafter\def\csname PY@tok@nd\endcsname{\def\PY@tc##1{\textcolor[rgb]{0.67,0.13,1.00}{##1}}}
\expandafter\def\csname PY@tok@ne\endcsname{\let\PY@bf=\textbf\def\PY@tc##1{\textcolor[rgb]{0.82,0.25,0.23}{##1}}}
\expandafter\def\csname PY@tok@nf\endcsname{\def\PY@tc##1{\textcolor[rgb]{0.00,0.00,1.00}{##1}}}
\expandafter\def\csname PY@tok@si\endcsname{\let\PY@bf=\textbf\def\PY@tc##1{\textcolor[rgb]{0.73,0.40,0.53}{##1}}}
\expandafter\def\csname PY@tok@s2\endcsname{\def\PY@tc##1{\textcolor[rgb]{0.73,0.13,0.13}{##1}}}
\expandafter\def\csname PY@tok@vi\endcsname{\def\PY@tc##1{\textcolor[rgb]{0.10,0.09,0.49}{##1}}}
\expandafter\def\csname PY@tok@nt\endcsname{\let\PY@bf=\textbf\def\PY@tc##1{\textcolor[rgb]{0.00,0.50,0.00}{##1}}}
\expandafter\def\csname PY@tok@nv\endcsname{\def\PY@tc##1{\textcolor[rgb]{0.10,0.09,0.49}{##1}}}
\expandafter\def\csname PY@tok@s1\endcsname{\def\PY@tc##1{\textcolor[rgb]{0.73,0.13,0.13}{##1}}}
\expandafter\def\csname PY@tok@sh\endcsname{\def\PY@tc##1{\textcolor[rgb]{0.73,0.13,0.13}{##1}}}
\expandafter\def\csname PY@tok@sc\endcsname{\def\PY@tc##1{\textcolor[rgb]{0.73,0.13,0.13}{##1}}}
\expandafter\def\csname PY@tok@sx\endcsname{\def\PY@tc##1{\textcolor[rgb]{0.00,0.50,0.00}{##1}}}
\expandafter\def\csname PY@tok@bp\endcsname{\def\PY@tc##1{\textcolor[rgb]{0.00,0.50,0.00}{##1}}}
\expandafter\def\csname PY@tok@c1\endcsname{\let\PY@it=\textit\def\PY@tc##1{\textcolor[rgb]{0.25,0.50,0.50}{##1}}}
\expandafter\def\csname PY@tok@kc\endcsname{\let\PY@bf=\textbf\def\PY@tc##1{\textcolor[rgb]{0.00,0.50,0.00}{##1}}}
\expandafter\def\csname PY@tok@c\endcsname{\let\PY@it=\textit\def\PY@tc##1{\textcolor[rgb]{0.25,0.50,0.50}{##1}}}
\expandafter\def\csname PY@tok@mf\endcsname{\def\PY@tc##1{\textcolor[rgb]{0.40,0.40,0.40}{##1}}}
\expandafter\def\csname PY@tok@err\endcsname{\def\PY@bc##1{\setlength{\fboxsep}{0pt}\fcolorbox[rgb]{1.00,0.00,0.00}{1,1,1}{\strut ##1}}}
\expandafter\def\csname PY@tok@kd\endcsname{\let\PY@bf=\textbf\def\PY@tc##1{\textcolor[rgb]{0.00,0.50,0.00}{##1}}}
\expandafter\def\csname PY@tok@ss\endcsname{\def\PY@tc##1{\textcolor[rgb]{0.10,0.09,0.49}{##1}}}
\expandafter\def\csname PY@tok@sr\endcsname{\def\PY@tc##1{\textcolor[rgb]{0.73,0.40,0.53}{##1}}}
\expandafter\def\csname PY@tok@mo\endcsname{\def\PY@tc##1{\textcolor[rgb]{0.40,0.40,0.40}{##1}}}
\expandafter\def\csname PY@tok@kn\endcsname{\let\PY@bf=\textbf\def\PY@tc##1{\textcolor[rgb]{0.00,0.50,0.00}{##1}}}
\expandafter\def\csname PY@tok@mi\endcsname{\def\PY@tc##1{\textcolor[rgb]{0.40,0.40,0.40}{##1}}}
\expandafter\def\csname PY@tok@gp\endcsname{\let\PY@bf=\textbf\def\PY@tc##1{\textcolor[rgb]{0.00,0.00,0.50}{##1}}}
\expandafter\def\csname PY@tok@o\endcsname{\def\PY@tc##1{\textcolor[rgb]{0.40,0.40,0.40}{##1}}}
\expandafter\def\csname PY@tok@kr\endcsname{\let\PY@bf=\textbf\def\PY@tc##1{\textcolor[rgb]{0.00,0.50,0.00}{##1}}}
\expandafter\def\csname PY@tok@s\endcsname{\def\PY@tc##1{\textcolor[rgb]{0.73,0.13,0.13}{##1}}}
\expandafter\def\csname PY@tok@kp\endcsname{\def\PY@tc##1{\textcolor[rgb]{0.00,0.50,0.00}{##1}}}
\expandafter\def\csname PY@tok@w\endcsname{\def\PY@tc##1{\textcolor[rgb]{0.73,0.73,0.73}{##1}}}
\expandafter\def\csname PY@tok@kt\endcsname{\def\PY@tc##1{\textcolor[rgb]{0.69,0.00,0.25}{##1}}}
\expandafter\def\csname PY@tok@ow\endcsname{\let\PY@bf=\textbf\def\PY@tc##1{\textcolor[rgb]{0.67,0.13,1.00}{##1}}}
\expandafter\def\csname PY@tok@sb\endcsname{\def\PY@tc##1{\textcolor[rgb]{0.73,0.13,0.13}{##1}}}
\expandafter\def\csname PY@tok@k\endcsname{\let\PY@bf=\textbf\def\PY@tc##1{\textcolor[rgb]{0.00,0.50,0.00}{##1}}}
\expandafter\def\csname PY@tok@se\endcsname{\let\PY@bf=\textbf\def\PY@tc##1{\textcolor[rgb]{0.73,0.40,0.13}{##1}}}
\expandafter\def\csname PY@tok@sd\endcsname{\let\PY@it=\textit\def\PY@tc##1{\textcolor[rgb]{0.73,0.13,0.13}{##1}}}

\def\PYZbs{\char`\\}
\def\PYZus{\char`\_}
\def\PYZob{\char`\{}
\def\PYZcb{\char`\}}
\def\PYZca{\char`\^}
\def\PYZam{\char`\&}
\def\PYZlt{\char`\<}
\def\PYZgt{\char`\>}
\def\PYZsh{\char`\#}
\def\PYZpc{\char`\%}
\def\PYZdl{\char`\$}
\def\PYZti{\char`\~}
% for compatibility with earlier versions
\def\PYZat{@}
\def\PYZlb{[}
\def\PYZrb{]}
\makeatother

\begin{document}
\LARGE

%-------------------------------------------------------------------------------
\center{Блоки кода}
\begin{itemize}
    \item Блоки ограничивают участок кода, принадлежащий управляющей конструкции
    \item Начинаются с “:”, которым оканчивается конструкция 
    \item Все строки блока имеют уровень отступа равным начальной строке блока
    \item Отступы делаются с помошью табуляции или пробелов
    \item Блоки могут содержать другие блоки (с более глубокими отступами)
\end{itemize}
\vspace{15pt}
\begin{lstlisting}
    Some_contruction:
        y = 2
        z = x + y
    #end_of_block
\end{lstlisting}
\newpage

%-------------------------------------------------------------------------------
\center{Блоки кода}
\begin{itemize}
    \item Блоки это не области видимости переменных. Переменные видны и после выхода из блока
    \item \lstinline$pass$ – пустой блок
\end{itemize}
\newpage

%-------------------------------------------------------------------------------
\center{if - Условное выполнение участков кода}
\vspace{15pt}
\begin{lstlisting}
    if  condition1 :
        pass # excuted if condition1 is true
    elif condition2 :
        pass # excuted if condition1 is false and condition2 is true
    #... 
    else:
        pass # executed if all conditions is false 
\end{lstlisting}
\newpage

%-------------------------------------------------------------------------------
\center{if}
\vspace{15pt}
\begin{lstlisting}
    x = 12
    sign = 0
    if x > 0:
        print x, "positive"
        sign = 1
    elif x < 0:
        print x, "negative"
        sign = -1
    else:
        print x, "== 0"
        sign = 0
\end{lstlisting}
\newpage

%-------------------------------------------------------------------------------
\center{inline if}
\vspace{15pt}
\begin{lstlisting}
    res = x if x >= 0 else -x
    # res = (x >= 0 ? x : -x)
\end{lstlisting}
\newpage

%-------------------------------------------------------------------------------
\center{while}
\vspace{15pt}
\begin{lstlisting}
    while condition:
        pass # executed while condition is true
    else:
        pass # if no error or break in body

    x = 1
    while x < 100:
        print x, "less than 100"
        x *= 2
\end{lstlisting}
\newpage

%-------------------------------------------------------------------------------
\center{for - цикл по множеству}
\vspace{15pt}
\begin{lstlisting}
    for x in iterable:
        func(x) # for each element in iterable
    else:
        pass # if no error or break in body

    sum = 0
    for x in range(100):
        sum += x
    print x  # 99 * 100 / 2

    for i in range(n): # xrange(n)
        pass

\end{lstlisting}
\newpage

%-------------------------------------------------------------------------------
\center{for undercover}
\vspace{15pt}
\begin{lstlisting}
    for x in container:
        f(x)

    # some times equal to

    _tmp = 0
    while _tmp < len(container):
        x = container[_tmp]
        f(x)
        _tmp += 1
\end{lstlisting}
\newpage

%-------------------------------------------------------------------------------
\center{break \& сontinue как всегда}
\begin{itemize}
    \item \lstinline!break!  выходит из цикла
    \item \lstinline!continue! переходит к следующей итерации
\end{itemize}
\newpage

%-------------------------------------------------------------------------------
\center{Задача}

\begin{itemize}
    \item нужно декодировать телефонный номера для АОН.
    \item По запросу АОНа АТС посылает телефонный номер, используя следующие правила:
    \item - Если цифра повторяется менее 2 раз, то она должна быть отброшена
    \item - Каждая значащая цифра повторяется минимум 2 раза
    \item - Если в номере идут несколько цифр подряд, то для обозначения «такая же цифра как
         предыдущая» используется идущий 2  или более подряд раз знак \#
    \item Входящая строка 4434\#\#\#552222311333661 => 4452136
\end{itemize}
\newpage

%-------------------------------------------------------------------------------
\center{list – Список (Массив)}
\begin{itemize}
    \item Упорядоченное множество элементов, доступ по номеру
    \item \lstinline!var = [1, 2, 3]!
    \item Индексация \lstinline!arr[x]!
    \item Срезы 
            \lstinline!arr[frm:to:step]! \\
            \lstinline![arr[frm], arr[frm + step], ....., ]!
    \item Отрицательный индекс - отсчет от конца. x[-1]
    \item Отсутвие индекса - frm -> 0, to -> -1, step -> 1
    \item arr[::-1] - инверсия элементов
    \item arr[:] - копия
\end{itemize}
\newpage

%-------------------------------------------------------------------------------
\center{list – Список (Массив)}
{
\Huge
\begin{flushleft}
$x = [0_{-6}^{0}, 1_{-5}^{1}, 2_{-4}^{2}, 3_{-3}^{3}, 4_{-2}^{4}, 5_{-1}^{5}]$ \\
\vspace{0.5cm}
x[2] == 2 \hspace{2cm}[\textcolor{red}{0, 1,} 2, \textcolor{red}{3, 4, 5}] \\
\vspace{0.5cm}
x[-2] == 4 \hspace{2cm}[\textcolor{red}{0, 1, 2, 3,} 4, \textcolor{red}{5}] \\
\vspace{0.5cm}
x[2:] == [2, 3, 4, 5] \hspace{2cm}[\textcolor{red}{0, 1, }2, 3, 4, 5] \\
\vspace{0.5cm}
x[-2:] == [4, 5] \hspace{2cm}[\textcolor{red}{0, 1, 2, 3,} 4, 5] \\
\vspace{0.5cm}
x[1:-1] == [1, 2, 3, 4] \hspace{2cm}[\textcolor{red}{0,} 1, 2, 3, 4, \textcolor{red}{5}] \\
\vspace{0.5cm}
x[1:-1:2] == [1, 3] \hspace{2cm}[\textcolor{red}{0,} 1, \textcolor{orange}{2,} 3, \textcolor{orange}{4,} \textcolor{red}{5}] \\
\vspace{0.5cm}
x[::-1] == [5, 4, 3, 2, 1, 0] \\
\end{flushleft}
}
\newpage

%-------------------------------------------------------------------------------
\center{list – Операции нам элементам и срезам}
\vspace{15pt}
\begin{lstlisting}
    x = [3, 4, 5, 6]
    x[::2] = [2, 2]  # x == [2, 4, 2, 6]
    x[::2] = 2 # error
    del x[1] # x == [3, 5, 6]
    x = [1, None, True, ["123", 2.4]]
    [1, 2, 3] + ["a", "b"] # [1, 2, 3, "a", "b"]
\end{lstlisting}
\newpage

%-------------------------------------------------------------------------------
\center{Методы списка}
{\Large
\vspace{15pt}
\begin{lstlisting}
    # arr.append(val)
    [1, 2].append(3) == [1, 2, 3]

    # arr.extend(arr2)
    [1, 2].extend([2, 3]) == [1, 2, 2, 3]

    # arr.pop()
    x = [1, 2]
    x.pop() == 2
    print x # [1]pfdnhf 

    # arr.insert(pos, val)
    [1, 2].insert(0, "abc") == ["abc", 1, 2]

    [1, 2].index(2) == 1
    [1, 2].reverse() == [2, 1]
    [1, 2, 4, 1, 2, 4, 1, 1].count(1) == 4
    x = [1, 3, 2]
    x.remove(1) # x == [3, 2]
    x.sort() # x == [2, 3]
\end{lstlisting}
}
\newpage

%-------------------------------------------------------------------------------
\center{Range}
\vspace{15pt}
\begin{lstlisting}
    range(x) == (0, ..., x – 1)
    range(x, y, z) == range(x)[:y:z]
\end{lstlisting}
\newpage

%-------------------------------------------------------------------------------
\center{assert}
\begin{itemize}
    \item \lstinline!assert expr[, msg]!
    \item \lstinline!assert x == 1, "X should be equal to 0"!
\end{itemize}
\newpage

%-------------------------------------------------------------------------------
\center{Функции - минимум}
\begin{lstlisting}
    def func_name1(param1, param2):
        "documentation"
        # block
        x = param1 + param2
        return x

    def func_name2(param1, param2):
        "documentation"
        # block
        x = param1 + param2
        if x > 0:
            return x
        else:
            return 0
\end{lstlisting}
\newpage

%-------------------------------------------------------------------------------
\center{Program template}
\begin{lstlisting}
    #!/usr/bin/end python
    # -*- coding:utf8 -*-
    ......

    def main():
        res = 0
        .....
        return res

    if __name__ == "__main__":
        exit(main())
\end{lstlisting}
\newpage


%-------------------------------------------------------------------------------
\center{tuple – кортеж}
\begin{itemize}
    \item Константный список ( но можно изменять элементы, если они не константные)
\end{itemize}
\vspace{15pt}
\begin{lstlisting}
    tpl = (1, 2)
    tpl = 1,2
    tpl[1] = 3 # error
    tpl = (1, [2, 3, 4])
    tpl[1].append(1) => (1, [2, 3, 4, 1])
    (1) == 1
    (1,) == (1,)
    user, passwd = ("user", "qwerty") 
\end{lstlisting}
\newpage

%-------------------------------------------------------------------------------
\center{dict - словарь}
\begin{itemize}
    \item Набор пар (ключ, значение), с быстрым поиском по ключу
    \lstinline$x = {1:2, "3":4}$
    \item Только константные ключи (tuple - ok)
    \item Элементы неупорядоченны
    \item Нет срезов
\end{itemize}
\vspace{15pt}
\begin{lstlisting}
    x[1] == 2
    x[2] #error
    1 in x == True
    x[17] = True
    # x = {1:2, "3":4, 17:True}
\end{lstlisting}
\newpage

%-------------------------------------------------------------------------------
\center{dict – Словарь}
\vspace{15pt}
\begin{lstlisting}
    x = {1:2, "3":"4"}
    dict(a=1, b=2) == {"a":1, "b":2}
    x.items() == [(1, 2), ("3", "4")]
    x.values() == [2, "4"]
    x.keys() == [1, "3"]
    x.copy() == {1:2, "3":"4"}
    x.setdefault(key, val) == val # if key not in x else x[key]
    x.get(5, None) == None # if 5 not in x else x[5]
    x.clear() # {}
    x.update(y)
    dict.fromkeys(keys, val) # {key[0]:val, key[1]:val, ..} default val is None
\end{lstlisting}
\newpage

%-------------------------------------------------------------------------------
\end{document}
