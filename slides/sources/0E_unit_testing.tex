% python classes slides - classes
% (c) 2014 Kostiantyn Danylov aka koder 
% koder.mail@gmail.com
% distributed under CC-BY licence
% http://creativecommons.org/licenses/by/3.0/deed.en

\documentclass{article}
% XeLaTeX
\usepackage{xltxtra}
\usepackage{xunicode}
\usepackage{listings}
\usepackage[landscape]{geometry}

% Fonts
\setmainfont{DejaVu Sans} %{Arial}
\newfontfamily\cyrillicfont{Nimbus Roman No9 L} %{Arial}
\setmonofont{Courier New}
%\setmonofont{Ubuntu Mono}

%\setmonofont{DejaVu Sans Mono}

% Lang
\usepackage{polyglossia}
\setmainlanguage{russian}
\setotherlanguage{english}
\usepackage[dvipsnames,table]{xcolor}


\ifx\pdfoutput\undefined
\usepackage{graphicx}
\else
\usepackage[pdftex]{graphicx}
\fi

\lstset{
	language=python,
	keywordstyle=\color{Emerald},%\texttt, 
	commentstyle=\color{OliveGreen},%\texttt,
	stringstyle=\color{Bittersweet},%\texttt,
	tabsize=4,
	numbers=left,
	xleftmargin=10pt,
	morekeywords={with,as},	
	numberstyle=\large,
	%identifierstyle=\texttt,
	%basicstyle=\texttt,
}

\usepackage{hyperref}

\hypersetup{
	colorlinks=true,
	urlcolor=blue
}

\usepackage{float}
%\floatstyle{boxed} 
%\restylefloat{figure}
\usepackage[normalem]{ulem}


\makeatletter
\def\PY@reset{\let\PY@it=\relax \let\PY@bf=\relax%
    \let\PY@ul=\relax \let\PY@tc=\relax%
    \let\PY@bc=\relax \let\PY@ff=\relax}
\def\PY@tok#1{\csname PY@tok@#1\endcsname}
\def\PY@toks#1+{\ifx\relax#1\empty\else%
    \PY@tok{#1}\expandafter\PY@toks\fi}
\def\PY@do#1{\PY@bc{\PY@tc{\PY@ul{%
    \PY@it{\PY@bf{\PY@ff{#1}}}}}}}
\def\PY#1#2{\PY@reset\PY@toks#1+\relax+\PY@do{#2}}

\expandafter\def\csname PY@tok@gd\endcsname{\def\PY@tc##1{\textcolor[rgb]{0.63,0.00,0.00}{##1}}}
\expandafter\def\csname PY@tok@gu\endcsname{\let\PY@bf=\textbf\def\PY@tc##1{\textcolor[rgb]{0.50,0.00,0.50}{##1}}}
\expandafter\def\csname PY@tok@gt\endcsname{\def\PY@tc##1{\textcolor[rgb]{0.00,0.25,0.82}{##1}}}
\expandafter\def\csname PY@tok@gs\endcsname{\let\PY@bf=\textbf}
\expandafter\def\csname PY@tok@gr\endcsname{\def\PY@tc##1{\textcolor[rgb]{1.00,0.00,0.00}{##1}}}
\expandafter\def\csname PY@tok@cm\endcsname{\let\PY@it=\textit\def\PY@tc##1{\textcolor[rgb]{0.25,0.50,0.50}{##1}}}
\expandafter\def\csname PY@tok@vg\endcsname{\def\PY@tc##1{\textcolor[rgb]{0.10,0.09,0.49}{##1}}}
\expandafter\def\csname PY@tok@m\endcsname{\def\PY@tc##1{\textcolor[rgb]{0.40,0.40,0.40}{##1}}}
\expandafter\def\csname PY@tok@mh\endcsname{\def\PY@tc##1{\textcolor[rgb]{0.40,0.40,0.40}{##1}}}
\expandafter\def\csname PY@tok@go\endcsname{\def\PY@tc##1{\textcolor[rgb]{0.50,0.50,0.50}{##1}}}
\expandafter\def\csname PY@tok@ge\endcsname{\let\PY@it=\textit}
\expandafter\def\csname PY@tok@vc\endcsname{\def\PY@tc##1{\textcolor[rgb]{0.10,0.09,0.49}{##1}}}
\expandafter\def\csname PY@tok@il\endcsname{\def\PY@tc##1{\textcolor[rgb]{0.40,0.40,0.40}{##1}}}
\expandafter\def\csname PY@tok@cs\endcsname{\let\PY@it=\textit\def\PY@tc##1{\textcolor[rgb]{0.25,0.50,0.50}{##1}}}
\expandafter\def\csname PY@tok@cp\endcsname{\def\PY@tc##1{\textcolor[rgb]{0.74,0.48,0.00}{##1}}}
\expandafter\def\csname PY@tok@gi\endcsname{\def\PY@tc##1{\textcolor[rgb]{0.00,0.63,0.00}{##1}}}
\expandafter\def\csname PY@tok@gh\endcsname{\let\PY@bf=\textbf\def\PY@tc##1{\textcolor[rgb]{0.00,0.00,0.50}{##1}}}
\expandafter\def\csname PY@tok@ni\endcsname{\let\PY@bf=\textbf\def\PY@tc##1{\textcolor[rgb]{0.60,0.60,0.60}{##1}}}
\expandafter\def\csname PY@tok@nl\endcsname{\def\PY@tc##1{\textcolor[rgb]{0.63,0.63,0.00}{##1}}}
\expandafter\def\csname PY@tok@nn\endcsname{\let\PY@bf=\textbf\def\PY@tc##1{\textcolor[rgb]{0.00,0.00,1.00}{##1}}}
\expandafter\def\csname PY@tok@no\endcsname{\def\PY@tc##1{\textcolor[rgb]{0.53,0.00,0.00}{##1}}}
\expandafter\def\csname PY@tok@na\endcsname{\def\PY@tc##1{\textcolor[rgb]{0.49,0.56,0.16}{##1}}}
\expandafter\def\csname PY@tok@nb\endcsname{\def\PY@tc##1{\textcolor[rgb]{0.00,0.50,0.00}{##1}}}
\expandafter\def\csname PY@tok@nc\endcsname{\let\PY@bf=\textbf\def\PY@tc##1{\textcolor[rgb]{0.00,0.00,1.00}{##1}}}
\expandafter\def\csname PY@tok@nd\endcsname{\def\PY@tc##1{\textcolor[rgb]{0.67,0.13,1.00}{##1}}}
\expandafter\def\csname PY@tok@ne\endcsname{\let\PY@bf=\textbf\def\PY@tc##1{\textcolor[rgb]{0.82,0.25,0.23}{##1}}}
\expandafter\def\csname PY@tok@nf\endcsname{\def\PY@tc##1{\textcolor[rgb]{0.00,0.00,1.00}{##1}}}
\expandafter\def\csname PY@tok@si\endcsname{\let\PY@bf=\textbf\def\PY@tc##1{\textcolor[rgb]{0.73,0.40,0.53}{##1}}}
\expandafter\def\csname PY@tok@s2\endcsname{\def\PY@tc##1{\textcolor[rgb]{0.73,0.13,0.13}{##1}}}
\expandafter\def\csname PY@tok@vi\endcsname{\def\PY@tc##1{\textcolor[rgb]{0.10,0.09,0.49}{##1}}}
\expandafter\def\csname PY@tok@nt\endcsname{\let\PY@bf=\textbf\def\PY@tc##1{\textcolor[rgb]{0.00,0.50,0.00}{##1}}}
\expandafter\def\csname PY@tok@nv\endcsname{\def\PY@tc##1{\textcolor[rgb]{0.10,0.09,0.49}{##1}}}
\expandafter\def\csname PY@tok@s1\endcsname{\def\PY@tc##1{\textcolor[rgb]{0.73,0.13,0.13}{##1}}}
\expandafter\def\csname PY@tok@sh\endcsname{\def\PY@tc##1{\textcolor[rgb]{0.73,0.13,0.13}{##1}}}
\expandafter\def\csname PY@tok@sc\endcsname{\def\PY@tc##1{\textcolor[rgb]{0.73,0.13,0.13}{##1}}}
\expandafter\def\csname PY@tok@sx\endcsname{\def\PY@tc##1{\textcolor[rgb]{0.00,0.50,0.00}{##1}}}
\expandafter\def\csname PY@tok@bp\endcsname{\def\PY@tc##1{\textcolor[rgb]{0.00,0.50,0.00}{##1}}}
\expandafter\def\csname PY@tok@c1\endcsname{\let\PY@it=\textit\def\PY@tc##1{\textcolor[rgb]{0.25,0.50,0.50}{##1}}}
\expandafter\def\csname PY@tok@kc\endcsname{\let\PY@bf=\textbf\def\PY@tc##1{\textcolor[rgb]{0.00,0.50,0.00}{##1}}}
\expandafter\def\csname PY@tok@c\endcsname{\let\PY@it=\textit\def\PY@tc##1{\textcolor[rgb]{0.25,0.50,0.50}{##1}}}
\expandafter\def\csname PY@tok@mf\endcsname{\def\PY@tc##1{\textcolor[rgb]{0.40,0.40,0.40}{##1}}}
\expandafter\def\csname PY@tok@err\endcsname{\def\PY@bc##1{\setlength{\fboxsep}{0pt}\fcolorbox[rgb]{1.00,0.00,0.00}{1,1,1}{\strut ##1}}}
\expandafter\def\csname PY@tok@kd\endcsname{\let\PY@bf=\textbf\def\PY@tc##1{\textcolor[rgb]{0.00,0.50,0.00}{##1}}}
\expandafter\def\csname PY@tok@ss\endcsname{\def\PY@tc##1{\textcolor[rgb]{0.10,0.09,0.49}{##1}}}
\expandafter\def\csname PY@tok@sr\endcsname{\def\PY@tc##1{\textcolor[rgb]{0.73,0.40,0.53}{##1}}}
\expandafter\def\csname PY@tok@mo\endcsname{\def\PY@tc##1{\textcolor[rgb]{0.40,0.40,0.40}{##1}}}
\expandafter\def\csname PY@tok@kn\endcsname{\let\PY@bf=\textbf\def\PY@tc##1{\textcolor[rgb]{0.00,0.50,0.00}{##1}}}
\expandafter\def\csname PY@tok@mi\endcsname{\def\PY@tc##1{\textcolor[rgb]{0.40,0.40,0.40}{##1}}}
\expandafter\def\csname PY@tok@gp\endcsname{\let\PY@bf=\textbf\def\PY@tc##1{\textcolor[rgb]{0.00,0.00,0.50}{##1}}}
\expandafter\def\csname PY@tok@o\endcsname{\def\PY@tc##1{\textcolor[rgb]{0.40,0.40,0.40}{##1}}}
\expandafter\def\csname PY@tok@kr\endcsname{\let\PY@bf=\textbf\def\PY@tc##1{\textcolor[rgb]{0.00,0.50,0.00}{##1}}}
\expandafter\def\csname PY@tok@s\endcsname{\def\PY@tc##1{\textcolor[rgb]{0.73,0.13,0.13}{##1}}}
\expandafter\def\csname PY@tok@kp\endcsname{\def\PY@tc##1{\textcolor[rgb]{0.00,0.50,0.00}{##1}}}
\expandafter\def\csname PY@tok@w\endcsname{\def\PY@tc##1{\textcolor[rgb]{0.73,0.73,0.73}{##1}}}
\expandafter\def\csname PY@tok@kt\endcsname{\def\PY@tc##1{\textcolor[rgb]{0.69,0.00,0.25}{##1}}}
\expandafter\def\csname PY@tok@ow\endcsname{\let\PY@bf=\textbf\def\PY@tc##1{\textcolor[rgb]{0.67,0.13,1.00}{##1}}}
\expandafter\def\csname PY@tok@sb\endcsname{\def\PY@tc##1{\textcolor[rgb]{0.73,0.13,0.13}{##1}}}
\expandafter\def\csname PY@tok@k\endcsname{\let\PY@bf=\textbf\def\PY@tc##1{\textcolor[rgb]{0.00,0.50,0.00}{##1}}}
\expandafter\def\csname PY@tok@se\endcsname{\let\PY@bf=\textbf\def\PY@tc##1{\textcolor[rgb]{0.73,0.40,0.13}{##1}}}
\expandafter\def\csname PY@tok@sd\endcsname{\let\PY@it=\textit\def\PY@tc##1{\textcolor[rgb]{0.73,0.13,0.13}{##1}}}

\def\PYZbs{\char`\\}
\def\PYZus{\char`\_}
\def\PYZob{\char`\{}
\def\PYZcb{\char`\}}
\def\PYZca{\char`\^}
\def\PYZam{\char`\&}
\def\PYZlt{\char`\<}
\def\PYZgt{\char`\>}
\def\PYZsh{\char`\#}
\def\PYZpc{\char`\%}
\def\PYZdl{\char`\$}
\def\PYZti{\char`\~}
% for compatibility with earlier versions
\def\PYZat{@}
\def\PYZlb{[}
\def\PYZrb{]}
\makeatother

\begin{document}
\LARGE

%-------------------------------------------------------------------------------
\begin{center}Юнит-тестирование\end{center}
\begin{itemize}
	\item http://pycheesecake.org/wiki/PythonTestingToolsTaxonomy
	\item Тестирование: юнит, интеграционное, функциональное, нагрузочное, стресс, ...
	\item smoke tests vs overnight tests
	\item Проверки
	\item Поиск и исполнение тестов
	\item Моки
	\item Генерация отчетов и интеграция с системами CI
\end{itemize}
\newpage

%-------------------------------------------------------------------------------
\begin{center}\lstinline!assert & AssertionError!\end{center}
\begin{itemize}
	\item AssertionError - lingua franca юнит-тестов
	\item \lstinline!assert expr[, msg] - assert x == 3!
\end{itemize}

\begin{verbatim}
	In [3]: x = 1
	In [4]: y = 2
	In [5]: assert x == y
	------------------------
	AssertionError          Traceback (most recent call last)
	<ipython-input-5-0578c5880ed0> in <module>()
	----> 1 assert x == y
	AssertionError: 
\end{verbatim}

\begin{lstlisting}
	def test_some_func():
		assert some_func(1) == 2
\end{lstlisting}

\newpage

%-------------------------------------------------------------------------------
\begin{center}unittest\end{center}
\begin{lstlisting}
	import unittest

	class TestSome(unittest.TestCase):
	    def test_simple(self):
	        x = 1
	        y = 2
	        self.assertEquals(x, y)

	    def test_simple2(self):
	        y = x = 1
	        self.assertEquals(x, y)

	    def test_simple3(self):
	        x = 1 / 0

	if __name__ == "__main__":
	    unittest.main()
\end{lstlisting}
\newpage

%-------------------------------------------------------------------------------
\begin{center}\lstinline!assert & AssertionError!\end{center}
\Large
\begin{verbatim}
	$ python ut_simplest.py
	F.E
	==========================================
	ERROR: test_simple3 (__main__.TestSome)
	------------------------------------------
	Traceback (most recent call last):
	  File "ut_simplest.py", line 16, in test_simple3
	    y = 1 / 0
	ZeroDivisionError: integer division or modulo by zero

	==========================================
	FAIL: test_simple (__main__.TestSome)
	------------------------------------------
	Traceback (most recent call last):
	  File "ut_simplest.py", line 7, in test_simple
	    self.assertEquals(x, y)
	AssertionError: 1 != 2

	------------------------------------------
	Ran 3 tests in 0.001s

	FAILED (failures=1, errors=1)
\end{verbatim}
\LARGE
\newpage

%-------------------------------------------------------------------------------
\begin{center}Тестовые методы\end{center}
\begin{itemize}
	\item \lstinline!self.assertEqual(x, y)!
	\item \lstinline!self.assertIn(x, set)!
	\item \lstinline!self.assertRaises(ExcClass, func, params)!, 
	\item \lstinline!self.assertAlmostEqual(x, y)!
	\item ....
\end{itemize}

\begin{lstlisting}
def r(x):
	return 1 / x

class TC(TestCase):
	def test_r():
		self.assertRaises(ZeroDivisionError, r, 0)
		self.assertEquals(r(1), 1)
\end{lstlisting}
\newpage

%-------------------------------------------------------------------------------
\begin{center}setUp + tearDown\end{center}
\begin{lstlisting}
	class TestSome(unittest.TestCase):
	    def setUp(self):
	        self.x = 1
	        self.y = 1

	    def tearDown(self):
	    	del self.x
	    	del self.y

	    def test_simple3(self):
	        self.assertEquals(self.x, self.y)
\end{lstlisting}
\newpage

%-------------------------------------------------------------------------------
\begin{center}Поиск тестов\end{center}
\begin{itemize}
	\item python -m unittest test\_module1 test\_module2
	\item python -m unittest test\_module.TestClass
	\item python -m unittest test\_module.TestClass.test\_method
	\item python -m unittest discover project\_directory '*\_test.py'
\end{itemize}
\newpage

%-------------------------------------------------------------------------------
\begin{center} \href{http://www.kuwata-lab.com/oktest/oktest-py_users-guide.html}{oktest} \end{center}
\begin{lstlisting}
	from oktest import ok

	def simple_test():
		ok(1) == 1
		ok([]).is_a(list)
		ok("/tmp/x.txt").is_file()
\end{lstlisting}
\newpage

%-------------------------------------------------------------------------------
\begin{center}nose\end{center}
\begin{itemize}
	\item Замена unittest
	\item Плагины
	\item coverage, parallel test, etc
	\item CI integration
\end{itemize}
\newpage

%-------------------------------------------------------------------------------
\begin{center} mock \end{center}
\begin{lstlisting}
	import os
	import glob

	def remove_tmp_files():
		for fname in glob.glob("/tmp/*.tmp"):
			os.unlink(fname)
\end{lstlisting}
\newpage

%-------------------------------------------------------------------------------
\begin{center}\end{center}
\begin{lstlisting}
	import mock

	def glob_mock(path):
		return ["/tmp/x.tmp", "/tmp/y.sh"]

	all_unlinked = []
	def unlink_mock(path):
		return all_unlinked.append(path)

	@mock.patch("os.unlink", unlink_mock)
	@mock.patch("glob.glob", glob_mock)
	def test_remove_tmp():
		remove_tmp_files()
		ok(all_unlinked) == ["/tmp/x.tmp"]
\end{lstlisting}
\newpage

%-------------------------------------------------------------------------------
% \begin{center}\end{center}
% \begin{itemize}
% 	\item
% \end{itemize}
% \newpage

% %-------------------------------------------------------------------------------
% \begin{center}\end{center}
% \begin{itemize}
% 	\item
% \end{itemize}
% \newpage

% %-------------------------------------------------------------------------------
% \begin{center}\end{center}
% \begin{itemize}
% 	\item
% \end{itemize}
% \newpage

% %-------------------------------------------------------------------------------
% \begin{center}\end{center}
% \begin{itemize}
% 	\item
% \end{itemize}
% \newpage

%-------------------------------------------------------------------------------
\end{document}








