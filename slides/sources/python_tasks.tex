% python classes slides - introduction
% (c) 2012 Kostiantyn Danylov aka koder 
% koder.mail@gmail.com
% distributed under CC-BY licence
% http://creativecommons.org/licenses/by/3.0/deed.en

\documentclass{article}
% XeLaTeX
\usepackage{xltxtra}
\usepackage{xunicode}
\usepackage{listings}
\usepackage[landscape]{geometry}

% Fonts
\setmainfont{DejaVu Sans} %{Arial}
\newfontfamily\cyrillicfont{Nimbus Roman No9 L} %{Arial}
\setmonofont{Courier New}
%\setmonofont{Ubuntu Mono}

%\setmonofont{DejaVu Sans Mono}

% Lang
\usepackage{polyglossia}
\setmainlanguage{russian}
\setotherlanguage{english}
\usepackage[dvipsnames,table]{xcolor}


\ifx\pdfoutput\undefined
\usepackage{graphicx}
\else
\usepackage[pdftex]{graphicx}
\fi

\lstset{
	language=python,
	keywordstyle=\color{Emerald},%\texttt, 
	commentstyle=\color{OliveGreen},%\texttt,
	stringstyle=\color{Bittersweet},%\texttt,
	tabsize=4,
	numbers=left,
	xleftmargin=10pt,
	morekeywords={with,as},	
	numberstyle=\large,
	%identifierstyle=\texttt,
	%basicstyle=\texttt,
}

\usepackage{hyperref}

\hypersetup{
	colorlinks=true,
	urlcolor=blue
}

\usepackage{float}
%\floatstyle{boxed} 
%\restylefloat{figure}
\usepackage[normalem]{ulem}


\makeatletter
\def\PY@reset{\let\PY@it=\relax \let\PY@bf=\relax%
    \let\PY@ul=\relax \let\PY@tc=\relax%
    \let\PY@bc=\relax \let\PY@ff=\relax}
\def\PY@tok#1{\csname PY@tok@#1\endcsname}
\def\PY@toks#1+{\ifx\relax#1\empty\else%
    \PY@tok{#1}\expandafter\PY@toks\fi}
\def\PY@do#1{\PY@bc{\PY@tc{\PY@ul{%
    \PY@it{\PY@bf{\PY@ff{#1}}}}}}}
\def\PY#1#2{\PY@reset\PY@toks#1+\relax+\PY@do{#2}}

\expandafter\def\csname PY@tok@gd\endcsname{\def\PY@tc##1{\textcolor[rgb]{0.63,0.00,0.00}{##1}}}
\expandafter\def\csname PY@tok@gu\endcsname{\let\PY@bf=\textbf\def\PY@tc##1{\textcolor[rgb]{0.50,0.00,0.50}{##1}}}
\expandafter\def\csname PY@tok@gt\endcsname{\def\PY@tc##1{\textcolor[rgb]{0.00,0.25,0.82}{##1}}}
\expandafter\def\csname PY@tok@gs\endcsname{\let\PY@bf=\textbf}
\expandafter\def\csname PY@tok@gr\endcsname{\def\PY@tc##1{\textcolor[rgb]{1.00,0.00,0.00}{##1}}}
\expandafter\def\csname PY@tok@cm\endcsname{\let\PY@it=\textit\def\PY@tc##1{\textcolor[rgb]{0.25,0.50,0.50}{##1}}}
\expandafter\def\csname PY@tok@vg\endcsname{\def\PY@tc##1{\textcolor[rgb]{0.10,0.09,0.49}{##1}}}
\expandafter\def\csname PY@tok@m\endcsname{\def\PY@tc##1{\textcolor[rgb]{0.40,0.40,0.40}{##1}}}
\expandafter\def\csname PY@tok@mh\endcsname{\def\PY@tc##1{\textcolor[rgb]{0.40,0.40,0.40}{##1}}}
\expandafter\def\csname PY@tok@go\endcsname{\def\PY@tc##1{\textcolor[rgb]{0.50,0.50,0.50}{##1}}}
\expandafter\def\csname PY@tok@ge\endcsname{\let\PY@it=\textit}
\expandafter\def\csname PY@tok@vc\endcsname{\def\PY@tc##1{\textcolor[rgb]{0.10,0.09,0.49}{##1}}}
\expandafter\def\csname PY@tok@il\endcsname{\def\PY@tc##1{\textcolor[rgb]{0.40,0.40,0.40}{##1}}}
\expandafter\def\csname PY@tok@cs\endcsname{\let\PY@it=\textit\def\PY@tc##1{\textcolor[rgb]{0.25,0.50,0.50}{##1}}}
\expandafter\def\csname PY@tok@cp\endcsname{\def\PY@tc##1{\textcolor[rgb]{0.74,0.48,0.00}{##1}}}
\expandafter\def\csname PY@tok@gi\endcsname{\def\PY@tc##1{\textcolor[rgb]{0.00,0.63,0.00}{##1}}}
\expandafter\def\csname PY@tok@gh\endcsname{\let\PY@bf=\textbf\def\PY@tc##1{\textcolor[rgb]{0.00,0.00,0.50}{##1}}}
\expandafter\def\csname PY@tok@ni\endcsname{\let\PY@bf=\textbf\def\PY@tc##1{\textcolor[rgb]{0.60,0.60,0.60}{##1}}}
\expandafter\def\csname PY@tok@nl\endcsname{\def\PY@tc##1{\textcolor[rgb]{0.63,0.63,0.00}{##1}}}
\expandafter\def\csname PY@tok@nn\endcsname{\let\PY@bf=\textbf\def\PY@tc##1{\textcolor[rgb]{0.00,0.00,1.00}{##1}}}
\expandafter\def\csname PY@tok@no\endcsname{\def\PY@tc##1{\textcolor[rgb]{0.53,0.00,0.00}{##1}}}
\expandafter\def\csname PY@tok@na\endcsname{\def\PY@tc##1{\textcolor[rgb]{0.49,0.56,0.16}{##1}}}
\expandafter\def\csname PY@tok@nb\endcsname{\def\PY@tc##1{\textcolor[rgb]{0.00,0.50,0.00}{##1}}}
\expandafter\def\csname PY@tok@nc\endcsname{\let\PY@bf=\textbf\def\PY@tc##1{\textcolor[rgb]{0.00,0.00,1.00}{##1}}}
\expandafter\def\csname PY@tok@nd\endcsname{\def\PY@tc##1{\textcolor[rgb]{0.67,0.13,1.00}{##1}}}
\expandafter\def\csname PY@tok@ne\endcsname{\let\PY@bf=\textbf\def\PY@tc##1{\textcolor[rgb]{0.82,0.25,0.23}{##1}}}
\expandafter\def\csname PY@tok@nf\endcsname{\def\PY@tc##1{\textcolor[rgb]{0.00,0.00,1.00}{##1}}}
\expandafter\def\csname PY@tok@si\endcsname{\let\PY@bf=\textbf\def\PY@tc##1{\textcolor[rgb]{0.73,0.40,0.53}{##1}}}
\expandafter\def\csname PY@tok@s2\endcsname{\def\PY@tc##1{\textcolor[rgb]{0.73,0.13,0.13}{##1}}}
\expandafter\def\csname PY@tok@vi\endcsname{\def\PY@tc##1{\textcolor[rgb]{0.10,0.09,0.49}{##1}}}
\expandafter\def\csname PY@tok@nt\endcsname{\let\PY@bf=\textbf\def\PY@tc##1{\textcolor[rgb]{0.00,0.50,0.00}{##1}}}
\expandafter\def\csname PY@tok@nv\endcsname{\def\PY@tc##1{\textcolor[rgb]{0.10,0.09,0.49}{##1}}}
\expandafter\def\csname PY@tok@s1\endcsname{\def\PY@tc##1{\textcolor[rgb]{0.73,0.13,0.13}{##1}}}
\expandafter\def\csname PY@tok@sh\endcsname{\def\PY@tc##1{\textcolor[rgb]{0.73,0.13,0.13}{##1}}}
\expandafter\def\csname PY@tok@sc\endcsname{\def\PY@tc##1{\textcolor[rgb]{0.73,0.13,0.13}{##1}}}
\expandafter\def\csname PY@tok@sx\endcsname{\def\PY@tc##1{\textcolor[rgb]{0.00,0.50,0.00}{##1}}}
\expandafter\def\csname PY@tok@bp\endcsname{\def\PY@tc##1{\textcolor[rgb]{0.00,0.50,0.00}{##1}}}
\expandafter\def\csname PY@tok@c1\endcsname{\let\PY@it=\textit\def\PY@tc##1{\textcolor[rgb]{0.25,0.50,0.50}{##1}}}
\expandafter\def\csname PY@tok@kc\endcsname{\let\PY@bf=\textbf\def\PY@tc##1{\textcolor[rgb]{0.00,0.50,0.00}{##1}}}
\expandafter\def\csname PY@tok@c\endcsname{\let\PY@it=\textit\def\PY@tc##1{\textcolor[rgb]{0.25,0.50,0.50}{##1}}}
\expandafter\def\csname PY@tok@mf\endcsname{\def\PY@tc##1{\textcolor[rgb]{0.40,0.40,0.40}{##1}}}
\expandafter\def\csname PY@tok@err\endcsname{\def\PY@bc##1{\setlength{\fboxsep}{0pt}\fcolorbox[rgb]{1.00,0.00,0.00}{1,1,1}{\strut ##1}}}
\expandafter\def\csname PY@tok@kd\endcsname{\let\PY@bf=\textbf\def\PY@tc##1{\textcolor[rgb]{0.00,0.50,0.00}{##1}}}
\expandafter\def\csname PY@tok@ss\endcsname{\def\PY@tc##1{\textcolor[rgb]{0.10,0.09,0.49}{##1}}}
\expandafter\def\csname PY@tok@sr\endcsname{\def\PY@tc##1{\textcolor[rgb]{0.73,0.40,0.53}{##1}}}
\expandafter\def\csname PY@tok@mo\endcsname{\def\PY@tc##1{\textcolor[rgb]{0.40,0.40,0.40}{##1}}}
\expandafter\def\csname PY@tok@kn\endcsname{\let\PY@bf=\textbf\def\PY@tc##1{\textcolor[rgb]{0.00,0.50,0.00}{##1}}}
\expandafter\def\csname PY@tok@mi\endcsname{\def\PY@tc##1{\textcolor[rgb]{0.40,0.40,0.40}{##1}}}
\expandafter\def\csname PY@tok@gp\endcsname{\let\PY@bf=\textbf\def\PY@tc##1{\textcolor[rgb]{0.00,0.00,0.50}{##1}}}
\expandafter\def\csname PY@tok@o\endcsname{\def\PY@tc##1{\textcolor[rgb]{0.40,0.40,0.40}{##1}}}
\expandafter\def\csname PY@tok@kr\endcsname{\let\PY@bf=\textbf\def\PY@tc##1{\textcolor[rgb]{0.00,0.50,0.00}{##1}}}
\expandafter\def\csname PY@tok@s\endcsname{\def\PY@tc##1{\textcolor[rgb]{0.73,0.13,0.13}{##1}}}
\expandafter\def\csname PY@tok@kp\endcsname{\def\PY@tc##1{\textcolor[rgb]{0.00,0.50,0.00}{##1}}}
\expandafter\def\csname PY@tok@w\endcsname{\def\PY@tc##1{\textcolor[rgb]{0.73,0.73,0.73}{##1}}}
\expandafter\def\csname PY@tok@kt\endcsname{\def\PY@tc##1{\textcolor[rgb]{0.69,0.00,0.25}{##1}}}
\expandafter\def\csname PY@tok@ow\endcsname{\let\PY@bf=\textbf\def\PY@tc##1{\textcolor[rgb]{0.67,0.13,1.00}{##1}}}
\expandafter\def\csname PY@tok@sb\endcsname{\def\PY@tc##1{\textcolor[rgb]{0.73,0.13,0.13}{##1}}}
\expandafter\def\csname PY@tok@k\endcsname{\let\PY@bf=\textbf\def\PY@tc##1{\textcolor[rgb]{0.00,0.50,0.00}{##1}}}
\expandafter\def\csname PY@tok@se\endcsname{\let\PY@bf=\textbf\def\PY@tc##1{\textcolor[rgb]{0.73,0.40,0.13}{##1}}}
\expandafter\def\csname PY@tok@sd\endcsname{\let\PY@it=\textit\def\PY@tc##1{\textcolor[rgb]{0.73,0.13,0.13}{##1}}}

\def\PYZbs{\char`\\}
\def\PYZus{\char`\_}
\def\PYZob{\char`\{}
\def\PYZcb{\char`\}}
\def\PYZca{\char`\^}
\def\PYZam{\char`\&}
\def\PYZlt{\char`\<}
\def\PYZgt{\char`\>}
\def\PYZsh{\char`\#}
\def\PYZpc{\char`\%}
\def\PYZdl{\char`\$}
\def\PYZti{\char`\~}
% for compatibility with earlier versions
\def\PYZat{@}
\def\PYZlb{[}
\def\PYZrb{]}
\makeatother

\begin{document}
\LARGE

%-------------------------------------------------------------------------------
{\center Разложить число на простые делители}
\newpage

%-------------------------------------------------------------------------------
{\center Декодирование АОН}
\begin{itemize}
    \item Нужно преобразовать строку по следующим правилам:
    \item Если символ идет 2 и больше раз подрят - записать его в результат 1 раз
    \item Если символ повторяется 1 раз - отбросить
    \item Если # повторяется два и более раз - последний символ, записанный в результт записать еще раз
    \item "" => ""
    \item "1" => ""
    \item "11" => "1"
    \item "11111" => "1"
    \item "11#" => "1"
    \item "11##" => "11"
    \item "11122234\#\#\#55" => "1225"
\end{itemize}
\newpage

%------------------------------------------------------------------------------
{\center Гномья сортировка}
Сравниваются соседние элементы. Если они неупорядоченны - они меняются местами и делается шаг назад.
Если они упорядоченны, то шаг вперед. Если дошли до конца, то сортировка оконченна.
\newpage

%------------------------------------------------------------------------------
{\center Двоичный поиск}
Найти елемент в упорядоченном массиве методом дихотомии.
\newpage

%------------------------------------------------------------------------------
{\center Кодирование Шеннона — Фано}
\begin{itemize}
    \item Метод выбора близкого к оптимальному кода для сообщения
    \item Сообщение бъется на элементы
    \item Изначально коды для всех элементов пустые
    \item Элементы множества выписывают в порядке убывания вероятностей.
    \item Множество делится на две части, суммарные вероятности символов которых
            максимально близки друг другу.
    \item К коду первыой половины элементов дописывается "0", второй "1"
    \item Алгоритм повторяется для обоих частей
    \item * Искать оптимальное разбиение
\end{itemize}
\newpage

9) кодирование и декодирование файла по Хаффману. 
На диске есть файл с именем "input.txt". Его нужно прочитать, зако-дировать символы 
использую алгоритм Хаффмана и записать результат в output.bin. В решении должно быть две функции 
hf_encode(string) str->str, и hf_decode(string) str->str. Первая кодирует, вторая деко-дирует. 
Входными элементами для алгоритма являются отдельные байты файла.

%------------------------------------------------------------------------------
{\center Интерпретатор minilisp}
Программа на mini-lisp имеет вид (oper param1 param2 para3 .... paramn),
здесь oper - имя функции - любой набор символов, кроме пробелов;
param2 - целое, строка в кавычках (без кавычек внутри) или другая программа на mini-lisp.
Допустимые oper - + (складывает все операнды), - (вычитает из первого все операнды), 
print (печатает все операнды через пробел). Нужно исполнить программу, переданную на вход.

\begin{itemize}
    \item (+ 1 2 3) => 6
    \item (print (+ "a" "bc")) => печатает abc
\end{itemize}
\newpage

%------------------------------------------------------------------------------

{\center интерпретатор подмножества языка forth}

Программа на Forth состоит из набора команд(слов), некоторые
из которых имеют параметры. Для хранения данных используется
стек - команды получают свои операнды с вершины стека и туда
же сохраняют результаты. В подмножестве 5 команд:

\begin{itemize}
    \item put значение - ложит значение на вершину стека. Значение может
быть числом или строкой. Строка заключается в кавычки, внутри
строки кавычек быть не может
    \item pop - убирает значение с вершины стека
    \item add - изымает из стека 2 значения, складывает их и заносит результат в стек
    \item sub - изымает из стека 2 значения, вычитает их и заносит результат в стек
    \item print - вынимает из стека 1 значение и печатает его.
\end{itemize}

Каждая команда начинается с новой строки. Строки начинающиеся с '\#' - комментарии. 
Программа должна содержать функцию eval_forth(), принимающую строку на языке forth и исполняющую ее. 
По умолчанию из main вызывать eval_forth("example.frt")

Например если в example.rft будет:

\begin{verbatim}
    put 1
    put 3
    25
    add
    print
\end{verbatim}

Программа должна напечатать '4'. Сложение имеет такой же
смысл, как и в питоне. Вычитание для строк не определено

\newpage

%------------------------------------------------------------------------------
{\center строковые функции}
Написать строковые функции xfind, xreplace, xsplit, xjoin используя
срезы строк (без применения других методов строк).
\begin{itemize}
    \item xfind(s1, s2) == s1.find(s2)
    \item xreplace(s1, s2, s3) == s1.replace(s2, s3)
    \item xsplit(s1, s2) == s1.split(s2)
    \item xjoin(s, array) == s.join(array)
\end{itemize}
\newpage

%------------------------------------------------------------------------------
{\center Умножение больших чисел}
Реализовать алгоритм Карацубы для умножения больших чисел.
$AB * CD == (A + B * 2^m) * (C + D * 2^m) 
         = A * C + 2^{2*m} * B * D + 2^m * (B * C + A * D)
         = A * C + 2^{2*m} * B * D + 2^m * ((A + B)(C + D) – A * C - B * D) $
\newpage

%------------------------------------------------------------------------------
{\center Алгоритм Кнута-Морриса-Пратта}
\begin{itemize}
    \item 
\end{itemize}
\newpage

%------------------------------------------------------------------------------
{\center Тетрис}

\newpage

%------------------------------------------------------------------------------
{\center Острова}
Задан двумерный массив из {0,1}. Островом называется связная группа единиц,
в т.ч. по диагонали. Посчитать количество островов.

\newpage

%------------------------------------------------------------------------------
\center{web crawler}
   
%------------------------------------------------------------------------------
% http://acm.mipt.ru/judge/problems.pl?psorto=compl~d&CGISESSID=660ae15a1fa989f66471c7b094375452
% http://codegolf.stackexchange.com/
% http://informatics.mccme.ru/moodle/ - много задач, но в основном слабые
