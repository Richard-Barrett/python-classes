% python classes slides - modules and packages
% (c) 2012 Kostiantyn Danylov aka koder 
% koder.mail@gmail.com
% distributed under CC-BY licence
% http://creativecommons.org/licenses/by/3.0/deed.en

\documentclass{article}
% XeLaTeX
\usepackage{xltxtra}
\usepackage{xunicode}
\usepackage{listings}
\usepackage[landscape]{geometry}

% Fonts
\setmainfont{DejaVu Sans} %{Arial}
\newfontfamily\cyrillicfont{Nimbus Roman No9 L} %{Arial}
\setmonofont{Courier New}
%\setmonofont{Ubuntu Mono}

%\setmonofont{DejaVu Sans Mono}

% Lang
\usepackage{polyglossia}
\setmainlanguage{russian}
\setotherlanguage{english}
\usepackage[dvipsnames,table]{xcolor}


\ifx\pdfoutput\undefined
\usepackage{graphicx}
\else
\usepackage[pdftex]{graphicx}
\fi

\lstset{
	language=python,
	keywordstyle=\color{Emerald},%\texttt, 
	commentstyle=\color{OliveGreen},%\texttt,
	stringstyle=\color{Bittersweet},%\texttt,
	tabsize=4,
	numbers=left,
	xleftmargin=10pt,
	morekeywords={with,as},	
	numberstyle=\large,
	%identifierstyle=\texttt,
	%basicstyle=\texttt,
}

\usepackage{hyperref}

\hypersetup{
	colorlinks=true,
	urlcolor=blue
}

\usepackage{float}
%\floatstyle{boxed} 
%\restylefloat{figure}
\usepackage[normalem]{ulem}


\makeatletter
\def\PY@reset{\let\PY@it=\relax \let\PY@bf=\relax%
    \let\PY@ul=\relax \let\PY@tc=\relax%
    \let\PY@bc=\relax \let\PY@ff=\relax}
\def\PY@tok#1{\csname PY@tok@#1\endcsname}
\def\PY@toks#1+{\ifx\relax#1\empty\else%
    \PY@tok{#1}\expandafter\PY@toks\fi}
\def\PY@do#1{\PY@bc{\PY@tc{\PY@ul{%
    \PY@it{\PY@bf{\PY@ff{#1}}}}}}}
\def\PY#1#2{\PY@reset\PY@toks#1+\relax+\PY@do{#2}}

\expandafter\def\csname PY@tok@gd\endcsname{\def\PY@tc##1{\textcolor[rgb]{0.63,0.00,0.00}{##1}}}
\expandafter\def\csname PY@tok@gu\endcsname{\let\PY@bf=\textbf\def\PY@tc##1{\textcolor[rgb]{0.50,0.00,0.50}{##1}}}
\expandafter\def\csname PY@tok@gt\endcsname{\def\PY@tc##1{\textcolor[rgb]{0.00,0.25,0.82}{##1}}}
\expandafter\def\csname PY@tok@gs\endcsname{\let\PY@bf=\textbf}
\expandafter\def\csname PY@tok@gr\endcsname{\def\PY@tc##1{\textcolor[rgb]{1.00,0.00,0.00}{##1}}}
\expandafter\def\csname PY@tok@cm\endcsname{\let\PY@it=\textit\def\PY@tc##1{\textcolor[rgb]{0.25,0.50,0.50}{##1}}}
\expandafter\def\csname PY@tok@vg\endcsname{\def\PY@tc##1{\textcolor[rgb]{0.10,0.09,0.49}{##1}}}
\expandafter\def\csname PY@tok@m\endcsname{\def\PY@tc##1{\textcolor[rgb]{0.40,0.40,0.40}{##1}}}
\expandafter\def\csname PY@tok@mh\endcsname{\def\PY@tc##1{\textcolor[rgb]{0.40,0.40,0.40}{##1}}}
\expandafter\def\csname PY@tok@go\endcsname{\def\PY@tc##1{\textcolor[rgb]{0.50,0.50,0.50}{##1}}}
\expandafter\def\csname PY@tok@ge\endcsname{\let\PY@it=\textit}
\expandafter\def\csname PY@tok@vc\endcsname{\def\PY@tc##1{\textcolor[rgb]{0.10,0.09,0.49}{##1}}}
\expandafter\def\csname PY@tok@il\endcsname{\def\PY@tc##1{\textcolor[rgb]{0.40,0.40,0.40}{##1}}}
\expandafter\def\csname PY@tok@cs\endcsname{\let\PY@it=\textit\def\PY@tc##1{\textcolor[rgb]{0.25,0.50,0.50}{##1}}}
\expandafter\def\csname PY@tok@cp\endcsname{\def\PY@tc##1{\textcolor[rgb]{0.74,0.48,0.00}{##1}}}
\expandafter\def\csname PY@tok@gi\endcsname{\def\PY@tc##1{\textcolor[rgb]{0.00,0.63,0.00}{##1}}}
\expandafter\def\csname PY@tok@gh\endcsname{\let\PY@bf=\textbf\def\PY@tc##1{\textcolor[rgb]{0.00,0.00,0.50}{##1}}}
\expandafter\def\csname PY@tok@ni\endcsname{\let\PY@bf=\textbf\def\PY@tc##1{\textcolor[rgb]{0.60,0.60,0.60}{##1}}}
\expandafter\def\csname PY@tok@nl\endcsname{\def\PY@tc##1{\textcolor[rgb]{0.63,0.63,0.00}{##1}}}
\expandafter\def\csname PY@tok@nn\endcsname{\let\PY@bf=\textbf\def\PY@tc##1{\textcolor[rgb]{0.00,0.00,1.00}{##1}}}
\expandafter\def\csname PY@tok@no\endcsname{\def\PY@tc##1{\textcolor[rgb]{0.53,0.00,0.00}{##1}}}
\expandafter\def\csname PY@tok@na\endcsname{\def\PY@tc##1{\textcolor[rgb]{0.49,0.56,0.16}{##1}}}
\expandafter\def\csname PY@tok@nb\endcsname{\def\PY@tc##1{\textcolor[rgb]{0.00,0.50,0.00}{##1}}}
\expandafter\def\csname PY@tok@nc\endcsname{\let\PY@bf=\textbf\def\PY@tc##1{\textcolor[rgb]{0.00,0.00,1.00}{##1}}}
\expandafter\def\csname PY@tok@nd\endcsname{\def\PY@tc##1{\textcolor[rgb]{0.67,0.13,1.00}{##1}}}
\expandafter\def\csname PY@tok@ne\endcsname{\let\PY@bf=\textbf\def\PY@tc##1{\textcolor[rgb]{0.82,0.25,0.23}{##1}}}
\expandafter\def\csname PY@tok@nf\endcsname{\def\PY@tc##1{\textcolor[rgb]{0.00,0.00,1.00}{##1}}}
\expandafter\def\csname PY@tok@si\endcsname{\let\PY@bf=\textbf\def\PY@tc##1{\textcolor[rgb]{0.73,0.40,0.53}{##1}}}
\expandafter\def\csname PY@tok@s2\endcsname{\def\PY@tc##1{\textcolor[rgb]{0.73,0.13,0.13}{##1}}}
\expandafter\def\csname PY@tok@vi\endcsname{\def\PY@tc##1{\textcolor[rgb]{0.10,0.09,0.49}{##1}}}
\expandafter\def\csname PY@tok@nt\endcsname{\let\PY@bf=\textbf\def\PY@tc##1{\textcolor[rgb]{0.00,0.50,0.00}{##1}}}
\expandafter\def\csname PY@tok@nv\endcsname{\def\PY@tc##1{\textcolor[rgb]{0.10,0.09,0.49}{##1}}}
\expandafter\def\csname PY@tok@s1\endcsname{\def\PY@tc##1{\textcolor[rgb]{0.73,0.13,0.13}{##1}}}
\expandafter\def\csname PY@tok@sh\endcsname{\def\PY@tc##1{\textcolor[rgb]{0.73,0.13,0.13}{##1}}}
\expandafter\def\csname PY@tok@sc\endcsname{\def\PY@tc##1{\textcolor[rgb]{0.73,0.13,0.13}{##1}}}
\expandafter\def\csname PY@tok@sx\endcsname{\def\PY@tc##1{\textcolor[rgb]{0.00,0.50,0.00}{##1}}}
\expandafter\def\csname PY@tok@bp\endcsname{\def\PY@tc##1{\textcolor[rgb]{0.00,0.50,0.00}{##1}}}
\expandafter\def\csname PY@tok@c1\endcsname{\let\PY@it=\textit\def\PY@tc##1{\textcolor[rgb]{0.25,0.50,0.50}{##1}}}
\expandafter\def\csname PY@tok@kc\endcsname{\let\PY@bf=\textbf\def\PY@tc##1{\textcolor[rgb]{0.00,0.50,0.00}{##1}}}
\expandafter\def\csname PY@tok@c\endcsname{\let\PY@it=\textit\def\PY@tc##1{\textcolor[rgb]{0.25,0.50,0.50}{##1}}}
\expandafter\def\csname PY@tok@mf\endcsname{\def\PY@tc##1{\textcolor[rgb]{0.40,0.40,0.40}{##1}}}
\expandafter\def\csname PY@tok@err\endcsname{\def\PY@bc##1{\setlength{\fboxsep}{0pt}\fcolorbox[rgb]{1.00,0.00,0.00}{1,1,1}{\strut ##1}}}
\expandafter\def\csname PY@tok@kd\endcsname{\let\PY@bf=\textbf\def\PY@tc##1{\textcolor[rgb]{0.00,0.50,0.00}{##1}}}
\expandafter\def\csname PY@tok@ss\endcsname{\def\PY@tc##1{\textcolor[rgb]{0.10,0.09,0.49}{##1}}}
\expandafter\def\csname PY@tok@sr\endcsname{\def\PY@tc##1{\textcolor[rgb]{0.73,0.40,0.53}{##1}}}
\expandafter\def\csname PY@tok@mo\endcsname{\def\PY@tc##1{\textcolor[rgb]{0.40,0.40,0.40}{##1}}}
\expandafter\def\csname PY@tok@kn\endcsname{\let\PY@bf=\textbf\def\PY@tc##1{\textcolor[rgb]{0.00,0.50,0.00}{##1}}}
\expandafter\def\csname PY@tok@mi\endcsname{\def\PY@tc##1{\textcolor[rgb]{0.40,0.40,0.40}{##1}}}
\expandafter\def\csname PY@tok@gp\endcsname{\let\PY@bf=\textbf\def\PY@tc##1{\textcolor[rgb]{0.00,0.00,0.50}{##1}}}
\expandafter\def\csname PY@tok@o\endcsname{\def\PY@tc##1{\textcolor[rgb]{0.40,0.40,0.40}{##1}}}
\expandafter\def\csname PY@tok@kr\endcsname{\let\PY@bf=\textbf\def\PY@tc##1{\textcolor[rgb]{0.00,0.50,0.00}{##1}}}
\expandafter\def\csname PY@tok@s\endcsname{\def\PY@tc##1{\textcolor[rgb]{0.73,0.13,0.13}{##1}}}
\expandafter\def\csname PY@tok@kp\endcsname{\def\PY@tc##1{\textcolor[rgb]{0.00,0.50,0.00}{##1}}}
\expandafter\def\csname PY@tok@w\endcsname{\def\PY@tc##1{\textcolor[rgb]{0.73,0.73,0.73}{##1}}}
\expandafter\def\csname PY@tok@kt\endcsname{\def\PY@tc##1{\textcolor[rgb]{0.69,0.00,0.25}{##1}}}
\expandafter\def\csname PY@tok@ow\endcsname{\let\PY@bf=\textbf\def\PY@tc##1{\textcolor[rgb]{0.67,0.13,1.00}{##1}}}
\expandafter\def\csname PY@tok@sb\endcsname{\def\PY@tc##1{\textcolor[rgb]{0.73,0.13,0.13}{##1}}}
\expandafter\def\csname PY@tok@k\endcsname{\let\PY@bf=\textbf\def\PY@tc##1{\textcolor[rgb]{0.00,0.50,0.00}{##1}}}
\expandafter\def\csname PY@tok@se\endcsname{\let\PY@bf=\textbf\def\PY@tc##1{\textcolor[rgb]{0.73,0.40,0.13}{##1}}}
\expandafter\def\csname PY@tok@sd\endcsname{\let\PY@it=\textit\def\PY@tc##1{\textcolor[rgb]{0.73,0.13,0.13}{##1}}}

\def\PYZbs{\char`\\}
\def\PYZus{\char`\_}
\def\PYZob{\char`\{}
\def\PYZcb{\char`\}}
\def\PYZca{\char`\^}
\def\PYZam{\char`\&}
\def\PYZlt{\char`\<}
\def\PYZgt{\char`\>}
\def\PYZsh{\char`\#}
\def\PYZpc{\char`\%}
\def\PYZdl{\char`\$}
\def\PYZti{\char`\~}
% for compatibility with earlier versions
\def\PYZat{@}
\def\PYZlb{[}
\def\PYZrb{]}
\makeatother


\begin{document}
\LARGE

%-------------------------------------------------------------------------------
\begin{center} Модули \end{center}
\begin{itemize}
    \item python файлы в папке с текущим модулем или в sys.path
    \item .py => \{.pyc, .pyo\}
    \item .pyo = .pyc - asserts
    \item .pyd / .so
    \item \lstinline!import mod_name!
    \item \lstinline!import mod_name as new_name!
    \item \lstinline!from mod_name import obj_name, obj_name2!
    \item \lstinline!from mod_name import obj_name as obj_name2!
    \item \lstinline!from mod_name import *! все кроме \_.* или все из \lstinline!__all__!
\end{itemize}
\newpage

%-------------------------------------------------------------------------------
\begin{center} Модули \end{center}
\begin{lstlisting}
    import mod_name 
    # ===
    mod_name = __import__('mod_name')

    import mod_name as new_name 
    # ===  
    new_name = __import__('mod_name')

    from mod_name import some_name 
    # ===  
    __mod_name = __import__('mod_name')
    some_name = getattr(__mod_name, 'some_name')
    del __mod_name
\end{lstlisting}
\newpage

%-------------------------------------------------------------------------------
\begin{center} Пример модуля - mod.py \end{center}
\begin{lstlisting}
    __all__ = ["some_val", "func"]
    
    some_val = 16
    internal_data = 1

    def func(x):
        return x + some_val
\end{lstlisting}

\begin{center} Использование \end{center}
\begin{lstlisting}
    from mod import some_val, func
    print func(1)
\end{lstlisting}
\newpage

%-------------------------------------------------------------------------------
\begin{center} Пакеты \end{center}
\Large
\begin{verbatim}
    xml/
        __init__.py
        parsers/
            __init__.py
            expat/
                __init__.py
                bindings.py
            ...
        dom/
            __init__.py
            minidom.py
            ...
        sax/
            __init__.py
            saxutils.py
            ...
        etree/
            __init__.py
            ElementTree.py
            ....
\end{verbatim}
\newpage
\LARGE

%-------------------------------------------------------------------------------
\begin{center} Пакеты \end{center}
\begin{itemize}
    \item \lstinline!import xml! => xml/\_\_init\_\_.py
    \item \lstinline!import xml.expat! => xml/expat/\_\_init\_\_.py
    \item \lstinline!import xml.expat! приводит к появлению имени 'xml' c атрибутом 'expat'
    \item \lstinline!import xml.etree.ElementTree! => xml/expat/ElementTree.py
    \item Пакет имеет приоритет над модулем при импорте
    \item \_\_init\_\_.py часто служит для внешнего API пакета
    %\item Циклические импорты
    %\item Новые пакеты в 3.3
    %\item package.\_\_path\_\_
\end{itemize}
\newpage

%-------------------------------------------------------------------------------
\begin{center} Пакеты - \_\_init\_\_.py \end{center}
\begin{lstlisting}
    # __init__.py
    from some_module1 import api_func_1
    from some_module2 import api_func_2
    from some_module3 import ApiClass
    # ...
\end{lstlisting}
\newpage

%-------------------------------------------------------------------------------
\begin{center} Циклические импорты \end{center}

%-------------------------------------------------------------------------------
\begin{center} sys.path \end{center}
\begin{itemize}
    \item Модифицируемый список всех папок, в которых происходит поиск модулей и пакетов
    \item Со старта наполняется из :
    \begin{itemize}
    	\item встроенных в питон при компиляции настроек
    	\item PYTHONPATH
        \item реестра
        \item xxx.pth
    \end{itemize}
    \item Может содержать не только директории, при этом python использует
          пользовательские загрузчики для модулей
    \item По умолчанию поддерживается импорт из zip архивов
\end{itemize}

\begin{lstlisting}
    import sys
    sys.path.append(SOME_DIR)
    import module_in_SOME_DIR
\end{lstlisting}
\newpage

%-------------------------------------------------------------------------------
\begin{center} sys.modules \end{center}
\begin{itemize}
    \item Словарь, содержащий отображения имени модуля на объект-модуль.
    \item import сначала ищет объект по имени в этом словаре, потом вызывает импорт
    \item \lstinline!mod = reload(mod)! перегружает модуль
    \item при этом все создается новый объект-модуль, но старый не удаляется
    \item уже созданные объекты остаются связанными со старым модулем
\end{itemize}
\begin{lstlisting}
    sys.modules = {
        # ...
        'ast': <module 'ast' from '/usr/lib/python2.7/ast.pyc'>,
         'atexit': <module 'atexit' from '/usr/lib/python2.7/atexit.pyc'>,
         'base64': <module 'base64' from '/usr/lib/python2.7/base64.pyc'>,
         'bdb': <module 'bdb' from '/usr/lib/python2.7/bdb.pyc'>,
         'binascii': <module 'binascii' (built-in)>,
         # ...
    }
\end{lstlisting}
\newpage  
    
%-------------------------------------------------------------------------------
\begin{center} \_\_import\_\_ \end{center}
\begin{lstlisting}
    fc = open(module_file_path).read()
    module_code = compile(fc, module_file, "exec")
    loc = {}
    eval(module_code, loc)
    sys.modules[module_name] = ModuleObject(loc)
    func = loc['func']
\end{lstlisting}
\newpage

%-------------------------------------------------------------------------------
\begin{center} Подмена модуля \end{center}
\begin{lstlisting}
    class SomeClass(object):
        def m(self):
            return 1

    sys.modules['mod'] = SomeClass()
    from mod import m

    print m() # 1
\end{lstlisting}
\newpage

%-------------------------------------------------------------------------------
\begin{center} Специальные атрибуты \end{center}
\begin{itemize}
    \item \_\_autor\_\_
    \item \_\_name\_\_
    \item \_\_doc\_\_
    \item \_\_file\_\_
    \item \_\_name\_\_
\end{itemize}

\begin{lstlisting}
    import sys

    def main(argv=None):
        argv = argv if args is not None else sys.argv
        ...
        return code

    if __name__ == "__main__":
        sys.exit(main())
\end{lstlisting}
\newpage

%-------------------------------------------------------------------------------
\begin{center} Исполнение модуля \end{center}
\begin{itemize}
    \item python -m ....
    \item python -m timeit -h
    \item python -m timeit -s "a=b=1" "a+b"
    \item исполн
\end{itemize}
\newpage

%-------------------------------------------------------------------------------
\end{document}








