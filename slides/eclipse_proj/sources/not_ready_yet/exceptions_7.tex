\documentclass{article}

% XeLaTeX
\usepackage{xltxtra}
\usepackage{xunicode}
\usepackage{listings}
\usepackage[landscape]{geometry}

% Fonts
\setmainfont{DejaVu Sans} %{Arial}
\newfontfamily\cyrillicfont{Nimbus Roman No9 L} %{Arial}
\setmonofont{Courier New}
%\setmonofont{Ubuntu Mono}

%\setmonofont{DejaVu Sans Mono}

% Lang
\usepackage{polyglossia}
\setmainlanguage{russian}
\setotherlanguage{english}
\usepackage[dvipsnames,table]{xcolor}


\ifx\pdfoutput\undefined
\usepackage{graphicx}
\else
\usepackage[pdftex]{graphicx}
\fi

\lstset{
	language=python,
	keywordstyle=\color{Emerald},%\texttt, 
	commentstyle=\color{OliveGreen},%\texttt,
	stringstyle=\color{Bittersweet},%\texttt,
	tabsize=4,
	numbers=left,
	xleftmargin=10pt,
	morekeywords={with,as},	
	numberstyle=\large,
	%identifierstyle=\texttt,
	%basicstyle=\texttt,
}

\usepackage{hyperref}

\hypersetup{
	colorlinks=true,
	urlcolor=blue
}

\usepackage{float}
%\floatstyle{boxed} 
%\restylefloat{figure}
\usepackage[normalem]{ulem}


\begin{document}
\LARGE

%-------------------------------------------------------------------------------

\center{Исключения}
\begin{itemize}
\item Исключение – это событие, после которого дальнейшее продолжение 
		работы в данной точке бессмысленно. По итогу такого события 
		генерируется объект-исключение, и исполнение передается обработчику 
		ошибок этого типа
\item Пример – деление на 0, выбрасывается ошибка ZeroDivisionError
\item Исключения помогают упростить код, убрав из него множество 
		проверок и значительно облегчить восстановление программы после сбоя
\item Исключения упрощают доставку информации о ошибке от той точке, 
		в которой она возникла к той точке где она может быть обработанны
\end{itemize}
\newpage


%-------------------------------------------------------------------------------

\center{Исключения}
{
\LARGE \vspace{15pt}
\begin{lstlisting}

	try:
		block1
	except tp2 as var2:
		block2
	except (tp3, tp4) as var3:
		block3
	else:
		block5
	finally:
		block4

\end{lstlisting}
}
\newpage

%-------------------------------------------------------------------------------
\center{Исключения}
{
\LARGE \vspace{15pt}
\begin{lstlisting}
	try:
		raise tp2("xxx")  # <<<<
	except tp2 as var2:
		block2            # <<<<
	except (tp3, tp4) as var3:
		block3
	else:
		block5
	finally:
		block4            # <<<<
\end{lstlisting}
}

\newpage

%-------------------------------------------------------------------------------
\center{Исключения}
{
\LARGE \vspace{15pt}
\begin{lstlisting}
	try:
		pass           # <<<
	except tp2 as var2:
		block2
	except (tp3, tp4) as var3:
		block3
	else:
		block5         # <<<
	finally:
		block4         # <<<
\end{lstlisting}
}

\newpage

%-------------------------------------------------------------------------------
\center{Исключения}
{
\LARGE \vspace{15pt}
\begin{lstlisting}
	def f1(t, d, x, y):
		if t – d  == 0:
		    return None
		else:
		    t1 = ((x + y) / (t - d))
		    if t1 == 0:
		        return None
		    else:
		        return 1 / ((x + y) / (t - d))

	def f2(t, d, x, y):
		try:
		    return 1 / ((x + y) / (t - d))
		except ZeroDivisionError:
		    return None

\end{lstlisting}
}

\newpage

%-------------------------------------------------------------------------------

\center{Исключения. raise}
\begin{itemize}
\item \lstinline!raise ExceptionType("Some message")! порождает исключение
\item \lstinline!ExceptionType! должно наследовать \lstinline!Exception!
\item \lstinline!raise! без параметров разрешено только в блоке except. 
	Оно повторно выбрасывает исключение, которое сейчас обрабатывается
\end{itemize}
{
\LARGE \vspace{15pt}
\begin{lstlisting}
	try:
		func()
	except Exception:
		print "func cause exception"
		raise
\end{lstlisting}
}

\newpage

%-------------------------------------------------------------------------------
\center{Исключения. traceback}
\begin{itemize}
\item В обработчике исключения \lstinline!sys.exc_info()! возвращает тройку
	(Тип исключения, Объект исключения, Состояние Стека)

{
\LARGE \vspace{15pt}
\begin{lstlisting}
	try:
	    raise ValueError("ddd")
	except Exception as x:
		tb = sys.exc_info()[2]

	print tb.tb_frame # <frame at 0x....>
	print tb.tb_frame.f_lineno # 4
	print tb.tb_frame.f_code.co_name # '<module>'
	print tb.tb_frame.f_code.co_filename 
		# '<ipython-input-7-492d537cf800>'
	print tb.tb_next # <frame at 0x....> or None
	del tb
\end{lstlisting}
}

\end{itemize}
\newpage
%-------------------------------------------------------------------------------


\end{document}

